\section{Durchführung}
\label{sec:Durchführung}
Für die nachfolgenden Versuche wird immer der gleiche Versuchsaufbau verwendet. Dieser
besteht hauptsächlich aus einer Kupfer-Röntgenröhre, einem LiF-Kristall und einem
Geiger-Müller-Zählrohr.
\\
Die Durchführung kann automatisch stattfinden; es müssen jedoch verschiedene Parameter
eingestellt werden. Die Beschleunigungsspannung soll bei $U_\text{B}=\SI{35}{kV}$ liegen,
der Emissionsstrom bei $I = \SI{1}{mA}$. 
Zuletzt muss überprüft werden, ob Blende und LiF-Kristall in
den Halterungen stecken und die Blende waagerecht ausgerichtet ist.

\subsection{Überprüfung der Bragg-Bedingung}
\label{sec:Überprüfung der Bragg-Bedingung}
Für die Überprüfung der Bragg-Bedingung wird der Kristall auf einen konstanten
Kristallwinkel von $\theta = 14^\circ$ eingestellt. Das Zählrohr soll im Bereich zwischen
$26^\circ$ und $30^\circ$ mit einem Winkelzuwachs von $\Delta \alpha = 0,1^\circ$ die
Röntgenstrahlung messen, mit einer Integrationszeit von $\Delta t = \SI{5}{\second}$.

\subsection{Analyse eines Emissionsspektrums}
\label{sec:Analyse eines Emissionsspektrums}
Es soll die erste Beugungsordnung ($n=1$) gemessen werden, dazu soll das Zählrohr im
Bereich $8^\circ \leq \theta \leq 25^\circ$ mit $\Delta \alpha = 0,1^\circ$ die Strahlung
messen. Als Integrationszeit werden hier $\Delta t = \SI{10}{\second}$ eingestellt.

\subsection{Analyse der Absorptionsspektren}
\label{sec:Analyse der Absorptionsspektren}
Im letzten Teil sollen sechs verschiedene Absorber (z.B. Zink, Gallium, ...) vor das
Zählrohr gesetzt werden und in einem geeigneten Intervall für $\theta$ in
$0.1^\circ$-Schritten gemessen werden. Die Integrationszeit für diesen Versuchsteil ist
$\Delta t = \SI{20}{s}$.

