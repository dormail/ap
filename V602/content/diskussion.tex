\section{Diskussion}
\label{sec:Diskussion}
Im ersten Teil der Versuchsauswertung wurde die Bragg-Bedingung überprüft (\autoref{sec:bragg}). Dabei wurde eine Abweichung vom Theorie-Wert von $\Delta \theta=0,2°$ bzw. $\Delta \theta_{rel}=0,72\%$ ermittelt, was im im akzeptablen Fehlerereich liegt. \newline 

Im dritten Teil der Auswertung wurden die Absorptionsspektren einiger Aborber untersucht. Die Messabweichungen für Zink ergeben sich mit den recherchierten Literaturwerten aus \autoref{tab:Glanz} zu:
\begin{align*}
    \Delta\text{E}&=0,05 \, \mathrm{keV} & \Delta\text{E}_\text{rel}&=0,52\%\\
    \Delta\theta&=0.1° & \Delta\theta_\text{rel}&=0.54\%\\
    \Delta\sigma&=0.04 & \Delta\sigma_\text{rel}&=1,12\%
\end{align*}
\noindent
Auch hier sind die Abweichungen also gering und in einem tolerablen Fehlerbereich.