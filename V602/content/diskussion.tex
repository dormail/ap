\section{Diskussion}
\label{sec:Diskussion}
Im ersten Teil der Versuchsauswertung wurde die Bragg-Bedingung überprüft (\autoref{sec:bragg}). Dabei wurde eine Abweichung vom Theorie-Wert von $\Delta \theta=0,2°$ bzw. $\Delta \theta_{rel}=0,72\%$ ermittelt, was im akzeptablen Fehlerereich liegt. \newline 
Die Ergebnisse aus dem zweiten Auswertungsteil sind besonders nah an den recherchierten Literaturwerten. So weicht das gemesssene Ergebnis für die Energie der $K_\alpha-$Linie nur um $\Delta \text{E}_\alpha=0.01 \text{keV}$ vom recherchierten Literaturwert (siehe \autoref{sec:vorb}) ab. Für die $K_\beta-$Linie stimmt das gemessene Ergebnis sogar mit dem recherchierten Literaturwert überein, also $\Delta \text{E}_\beta=0$. Es konnten also sehr gute Messwerte aufgenommen werden. \newline
Im dritten Teil der Auswertung wurden die Absorptionsspektren einiger Aborber untersucht. Die Messabweichungen für Zink ergeben sich mit den recherchierten Literaturwerten aus \autoref{tab:Glanz} zu:
\begin{align*}
    \Delta\text{E}&=0,05 \, \mathrm{keV} & \Delta\text{E}_\text{rel}&=0,52\%\\
    \Delta\theta&=0.1° & \Delta\theta_\text{rel}&=0.54\%\\
    \Delta\sigma&=0.04 & \Delta\sigma_\text{rel}&=1,12\%
\end{align*}
\noindent
Für die Abschirmkonstanten der weiteren Absorber ergibt sich:
\begin{align*}
\Delta\sigma_{Gallium}&=0\\
\Delta\sigma_{Brom}&=0.05\\
\Delta\sigma_{Rubidium}&=0.1\\
\Delta\sigma_{Strontium}&=0.12\\
\end{align*}
Auch hier sind die Abweichungen also gering und in einem tolerablen Fehlerbereich.\newline
Im letzten Teil der Auswertung wurde die Rydbergkonstante aus den Messwerten und dem moseleyschen Gesetz zu $R = \SI{3.09e15}{\hertz}$ bestimmt. Im Vergleich zum Literaturwert, $R_{Lit}=\SI{3.29e15}{\hertz}$\cite{q}, erhält man eine Abweichung von $\Delta R =\SI{0.02e15}{\hertz}$ bzw. $\Delta R =0.61\%$. Somit konnte auch für die Rydbergkonstante experimentell ein Wert ermittelt werden der sehr nah am Literaturwert liegt.