\section{Vorbereitung}
\label{sec:vorb}
Zur Vorbereitung sollten die Energien, bei denen die Cu-$K_\alpha$ und Cu-$K_\beta$
erwartet werden, ermittelt werden. Diese liegen bei $E_{K_\alpha} = \SI{8.06}{\kilo\eV}$ und 
$E_{K_\beta} = \SI{8.92}{\kilo\eV}$ \cite{ld-didactic.de}. Mit der dadraus abgeleiteten Wellenlänge folgt aus
\autoref{eqn:braggsche-bedingung} die Glanzwinkel $\theta_{K_\alpha} = \SI{22.39}{\degree}$ 
und $\theta_{K_\beta} = \SI{20.13}{\degree}$.
% in content/glanzwinkel-theo.py ist ein Skript was die theta berechnet
Ferner sollte der Glanzwinkel zu verschiedenen Elementen ermittelt 
werden. Dieser kann mit \autoref{eqn:braggsche-bedingung}, den gegebenen Literaturwerten 
zur K-Kante $E_\text{K}^\text{Lit}$ \cite{wissen} und der Gitterkonstante 
$d = \SI{201.4}{\pico\meter}$ berechnet werden mit 
\begin{equation}
	\theta_\text{glanz} = \text{arcsin}\left(\frac{h \cdot c}{E \cdot 2d}\right) \, .
	\label{eqn:theta}
\end{equation}
Daraus folgen die Werte die in der \autoref{tab:Glanz} aufgeführt sind. 
\begin{table}
	\centering
	\caption{Literaturwerte und daraus errechnete Größen verschiedener Elemente}
	\label{tab:Glanz}
	\sisetup{table-format=2.1}
	\begin{tabular}{c c c c c}
		\toprule
		$ $ & $Z$ & $E_\text{K}^\text{Lit} \,/\, \si{\kilo\eV}$
		    & $\theta_\text{glanz}^\text{Lit} \,/\, \si{\degree}$ & 
		    $\sigma_\text{k}$\\
		    \midrule 
		Zn & 30 &  9,65 & 18.60042993 & 3,56 \\
		Ge & 32 & 11,10 & 16.09910447 & 3,68 \\
		Br & 35 & 13,47 & 13.20934876 & 3,85 \\
		Rb & 37 & 15,2 & 11.68329037 & 3,95 \\
		Sr & 38 & 16,10 & 11.02175704 & 4,01 \\
		Zr & 40 & 17,99 &  9.851577763 & 4,11 \\
		\bottomrule
	\end{tabular}
\end{table}

