\section*{Zielsetzung}
\label{sec:zielsetzung}

Das Ziel dieses Versuchs ist die Bestimmung der Halbwertszeiten eines Isotops und eines Isotopengemischs. 

\section{Theorie}
\label{sec:Theorie}


\subsection{Kernreaktionen}
\label{sec:a}
Wenn ein Neutron in einen Kern A eindringt, entsteht durch Absorption ein neuer Kern A*, der auch als Zwischenkern oder Compoundkern bezeichnet wird. Der Kern A* ist aber oft nicht in der Lage das aufgenommene Neutron oder ein anderes Nukleon abzustoßen, sodass er durch die Emission eines $\gamma$-Quants wieder in seinen Grundzustand übergeht. Dieses Verhalten lässt sich mit \autoref{eqn:a} beschreiben. 
\begin{equation}
    \label{eqn:a}
    {^{m}_{z}A} + {^{1}_{0}n} \rightarrow {^{m+1}_{z}A*} \rightarrow {^{m+1}_{z}A} + \gamma
\end{equation}


Der neu entstandene Kern ${^{m+1}_{z}A}$ ist in der Regel nicht stabil, da er mehr Neutronen enthält als ein entsprechender stabiler Kern. Die überschüssige Masse wird nach der Einsteinschen Beziehung in kinetische Energie eines Elektrons und Antineutrinos umgewandelt.
\\


\subsection{Erzeugung niederenergetischer Neutronen}
\label{sec:erzeugung}
Da Neutronen als freie Teilchen instabil sind, kommen sie in der Natur nicht vor und müssen für das vorliegende Experiment erzeugt werden. Die Neutronen wurden hier durch Beschuss von Be-Kernen mit $\alpha$-Teilchen gemäß \autoref{eqn:b} freigesetzt.
\begin{equation}
    \label{eqn:b}
    {^{9}_{4}Be} + {^{4}_{2}\alpha} \rightarrow {^{12}_{6}C}+{^{0}_{1}n}
\end{equation}

Zur Abbremsung der Neutronen werden sie durch Materieschichten mit leichten Kernen geschickt. Mit elastischen Stöße geben die Neutronen ihre Energie an
die leichten Kerne nach  
\begin{equation}
	E_\text{ü} = E_0 \frac{4Mm}{(M+m)^2}
		\label{eqn:e}
\end{equation}
ab. Die beste Substanz ist hier nach also Wasserstoff(geringste Massenunterschiede). Es wird daher ein Mantel aus Paraffin für die Neutronenquelle benutzt.

\subsection{Zerfall instabiler Isotope}
\label{sec:zerfall}
Stabile Isotope können durch Neutronenstrahlung in instabile verwandelt werden. Für Vanadium läuft die Aktivierung und der Zerfall gemäß folgender Gleichung ab:
\begin{equation}
    \label{eqn:v}
    {^{51}_{23}V} + n \rightarrow {^{52}_{23}V} \rightarrow {^{52}_{24}Cr} + \beta^- +v_e
\end{equation}

Die Zahl $N(t)$, der zu einem Zeitpunkt t noch nicht zerfallenen Kerne kann durch
\begin{equation}
    N(t) = N_0 exp(− \lambda t)
\end{equation}
bestimmt werden.
Die Halbwertszeit T kann dann durch
\begin{equation}
T = ln \frac{2}{\lambda}
\end{equation}
berechnet werden.

