\section{Diskussion}
\label{sec:Diskussion}

In diesem Versuch wurden drei Halbwertszeiten bestimmt:
\begin{enumerate}
	\item $(219 \pm 11) \si{\second}$ für Vanadium-52 (Ungefähr 3 Minuten und $(39 \pm 11)\si{\second}$),
	\item $(230 \pm 50) \si{\second}$ für Rhodium-104 (Ungefähr 3 Minuten und $(50 \pm 50)\si{\second}$),
	\item $(48 \pm 1.2) \si{\second}$ für Rhodium-104i.
\end{enumerate}
\noindent
Für diese Isotope lauten die Literaturwerte
\begin{enumerate}
	\item Vanadium-52: 3 Minuten 45 Sekunden \cite{chemie-vanadium},
	\item Rhodium-104: 4 Minuten 20 Sekunden \cite{internetchemie-rhodium},
	\item Rhodium-104i: $42,3$ Sekunden \cite{internetchemie-rhodium}.
\end{enumerate}
Damit lauten die relativen Abweichungen
\begin{itemize}
	\item Vanadium-52: $2.6\%$,
	\item Rhoidum-104: $10.15\%$,
	\item Rhoidum-104i: $14.22\%$.
\end{itemize}

\noindent
Mit nur $2.6\%$ Abweichung vom Literaturwert wurde die Halbwertszeit von Vanadium sehr genau bestimmt. Die 
Ergebnisse der zweiten Messung sind deutlich schlechter. Während der Literaturwert für die Halbwertszeit
von Rhodium-104 noch in der angegebenen Messunsicherheit ist, ist das bei dem Ergebnis für Rhodium-104i nicht
der Fall, d.h. dass dort ein für das zugrunde liegende Modell ein guter Wert gefunden wurde, aber dieses Modell
problematisch ist.
\\
Eine grundsätzliche Fehlerquelle bei der Messung vom Rhodium ist die, dass für die einzelnen Zeiten (schneller
Zerfall, langsamer Zerfall) weniger Datenpunkte verfügbar waren. Während für Vanadium noch 40 Messpunkte
aufgenommen wurden, lagen für den schnellen Zerfall von Rhodium-104i nur acht verwendbar Messpunkte vor.
Dies könnte man durch mehrfache Messung des 120-Sekunden-Zeitfensters, wo der schnelle Zerfall dominiert, 
ausbessern.
\\
Eine Fehlerquelle speziell für die Berechnung der Halbwertszeit von Rhodium-104i ist, 
dass die tatsächliche Zerfallsrate
auf dem Wert für Rhodium-104 beruht, welche aber auch schon $10.15\%$ Abweichung besitzt, sodass dort 
eine hohe Fortpflanzung der Abweichung vorliegt.
