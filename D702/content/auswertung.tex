\section{Auswertung}
\label{sec:Auswertung}

\subsection{Bestimmung der Untergrundrate}
\label{sec:aus:untergrundrate}

Für eine exakte Auswertung ist die Bestimmung der Nullrate notwendig.
In sieben Messintervallen mit $t = \SI{300}{\second}$ wurden die 
Untergrundraten aus \autoref{tab:untergrundrate} aufgenommen.
\begin{table}
	\centering
	\caption{Messdaten zur Untergrundrate, Messintervall jeweils $t = \SI{300}{\second}$.}
	\label{tab:untergrundrate}
	\begin{tabular}{c c}
		\toprule
		$N_U$ & $\Delta N_U$ \\
		\midrule
		129 & 11.36 \\
		143 & 11.96 \\
		144 & 12 \\
		136 & 11.66 \\
		139 & 11.79 \\
		126 & 11.22 \\
		158 & 12.57 \\
		\bottomrule
	\end{tabular}
\end{table}

\noindent
Da die Zählraten Poisson verteilt sind, ist die Messunsicherheit durch $\Delta N = \sqrt{N}$ 
gegeben.
\\
Zu den sieben Messdaten lautet der Durchschnitt
\begin{equation}
	\langle N_U \rangle_{t = \SI{300}{\second}} = 139 \pm 4.
\end{equation}
Wobei dieser im folgenden auf die verschiedenen Messintervalle angepasst werden muss.
\noindent
Die Abweichung folgt aus der linearen Approximation der Fehlerfortplanzung
\begin{equation}
	\Delta N_U = \sqrt{(\partial_x N)^2 (\Delta x)^2 + (\partial_y N)^2 (\Delta y)^2 + \hdots}.
	\label{eqn:fehlerfortpflanzung}
\end{equation}

\subsection{Zerfallskurve und Halbwertszeit von Vanadium}
\label{sec:aus:vanadium}

In dieser Messreihe wurde der Zerfall von Vanadium-52 betrachtet. Als Messintervall wurden hier
$\SI{30}{\second}$ gewählt, daher wird von den Messdaten der Untergrund
\begin{equation}
	\langle N_U \rangle_{t=\SI{30}{\second}} = 
	\frac{\langle N_U \rangle_{t=\SI{30}{\second}}}{10} = 14 \pm 0.4
\end{equation}
abgezogen.

\begin{longtable}{c c c c}
	%\centering
	\caption{Messdaten zum Zerfall von Vanadium-52. Die Abweichung folgt aus der Poisson Verteilung
		mit $\Delta N = \sqrt{N}$. Für die Korrigierte Zerfallsraten wurde der Untergrund 
	$14 \frac{\text{Zerfällen}}{\SI{30}{\second}}$ abgezogen.}\\
	%\label{tab:vanadium}
		\hline
		$t / \si{\second}$ & $N$ & $\Delta N$ &$N$ Korrigiert \\
		%\midrule
		\hline
		\endhead
		\hline
		\endfoot
		30   	&      189  	& 14.0 &      175.0 \\
		  60   	&      197  	& 14.0 &      183.0 \\
		  90   	&      150  	& 12.0 &      136.0 \\
		 120   	&      159  	& 13.0 &      145.0 \\
		 150   	&      155  	& 12.0 &      141.0 \\
		 180   	&      132  	& 11.0 &      118.0 \\
		 210   	&      117  	& 11.0 &      103.0 \\
		 240   	&      107  	& 10.0 &       93.0 \\
		 270   	&       94  	& 10.0 &       80.0 \\
		 300   	&      100  	& 10.0 &       86.0 \\
		 330   	&       79  	&  9.0 &       65.0 \\
		 360   	&       69  	&  8.0 &       55.0 \\
		 390   	&       81  	&  9.0 &       67.0 \\
		 420   	&       46  	&  7.0 &       32.0 \\
		 450   	&       49  	&  7.0 &       35.0 \\
		 480   	&       61  	&  8.0 &       47.0 \\
		 510   	&       56  	&  7.0 &       42.0 \\
		 540   	&       40  	&  6.0 &       26.0 \\
		 570   	&       45  	&  7.0 &       31.0 \\
		 600   	&       32  	&  6.0 &       18.0 \\
		 630   	&       27  	&  5.0 &       13.0 \\
		 660   	&       43  	&  7.0 &       29.0 \\
		 690   	&       35  	&  6.0 &       21.0 \\
		 720   	&       19  	&  4.0 &        5.0 \\
		 750   	&       28  	&  5.0 &       14.0 \\
		 780   	&       27  	&  5.0 &       13.0 \\
		 810   	&       36  	&  6.0 &       22.0 \\
		 840   	&       25  	&  5.0 &       11.0 \\
		 870   	&       29  	&  5.0 &       15.0 \\
		 900   	&       18  	&  4.0 &        4.0 \\
		 930   	&       17  	&  4.0 &        3.0 \\
		 960   	&       24  	&  5.0 &       10.0 \\
		 990   	&       21  	&  5.0 &        7.0 \\
		1020   	&       25  	&  5.0 &       11.0 \\
		1050   	&       21  	&  5.0 &        7.0 \\
		1080   	&       24  	&  5.0 &       10.0 \\
		1110   	&       25  	&  5.0 &       11.0 \\
		1140   	&       17  	&  4.0 &        3.0 \\
		1170   	&       20  	&  4.0 &        6.0 \\
		1200   	&       19  	&  4.0 &        5.0 \\
		1230   	&       20  	&  4.0 &        6.0 \\
		1260   	&       18  	&  4.0 &        4.0 \\
		1290   	&       16  	&  4.0 &        2.0 \\
		1320   	&       17  	&  4.0 &        3.0 \\
\end{longtable}

Für die im Zeitinterval $\Delta t$ gemessenen Zerfälle gilt
\begin{equation}
	\ln N_{\Delta t}(t) = \ln(N_0 \cdot (1 - e^{-\lambda \Delta t})) - \lambda t.
	\label{eqn:logarithmisch}
\end{equation}
\noindent
In einer linearen Ausgleichsrechnung wurde die Funktion 
\begin{equation}
	f(x) = m \cdot x + b
	\label{eqn:mx+b}
\end{equation}
\noindent
an die Messdaten angepasst. Dabei gilt zwischen \autoref{eqn:logarithmisch} und \autoref{eqn:mx+b} der 
Zusammenhang $m = -\lambda$.
\\
Die lineare Ausgleichsrechnung ergab
\begin{equation}
	\lambda = (3.2 \pm 0.16) \ \si{\second}^{-1}.
\end{equation}
\noindent Dann folgt die Halbwertszeit
\begin{equation}
	T = \SI{219\pm11}{\second}.
\end{equation}
Hier wurde wie bei der Untergrundrate mit der gaußschen Fehlerfortplanzung gerechnet (Siehe 
\autoref{eqn:fehlerfortpflanzung}).
\\
In \autoref{fig:zerfallskurve} sind die Messdaten und die Ausgleichsfunktion zu sehen.

\begin{figure}[H]
	\centering
	\includegraphics{build/vanadium.pdf}
	\caption{Messdaten und Fit für die Zerfallskurve von Vanadium.}
	\label{fig:zerfallskurve}
\end{figure}
