\section{Auswertung}
\label{sec:Auswertung}

\subsection{Bestimmung der Untergrundrate}
\label{sec:aus:untergrundrate}

Für eine exakte Auswertung ist die Bestimmung der Nullrate notwendig.
In sieben Messintervallen mit $t = \SI{300}{\second}$ wurden die 
Untergrundraten aus \autoref{tab:untergrundrate} aufgenommen.
\begin{table}
	\centering
	\caption{Messdaten zur Untergrundrate, Messintervall jeweils $t = \SI{300}{\second}$.}
	\label{tab:untergrundrate}
	\begin{tabular}{c c}
		\toprule
		$N_U$ & $\Delta N_U$ \\
		\midrule
		129 & 11,36 \\
		143 & 11,96 \\
		144 & 12 \\
		136 & 11,66 \\
		139 & 11,79 \\
		126 & 11,22 \\
		158 & 12,57 \\
		\bottomrule
	\end{tabular}
\end{table}

\noindent
Da die Zählraten Poisson-verteilt sind, ist die Messunsicherheit durch $\Delta N = \sqrt{N}$ 
gegeben. \\
Zu den sieben Messdaten lautet der Mittelwert
\begin{equation}
	\langle N_U \rangle_{t = \SI{300}{\second}} = 139 \pm 4.
\end{equation}
Wobei dieser im Folgenden auf die verschiedenen Messintervalle angepasst werden muss.
\noindent
Die Abweichung folgt aus der linearen Approximation der Fehlerfortplanzung
\begin{equation}
	\Delta N_U = \sqrt{(\partial_x N)^2 (\Delta x)^2 + (\partial_y N)^2 (\Delta y)^2 + \hdots}.
	\label{eqn:fehlerfortpflanzung}
\end{equation}

\subsection{Zerfallskurve und Halbwertszeit von Vanadium-52}
\label{sec:aus:vanadium}

In dieser Messreihe wurde der Zerfall von Vanadium-52 betrachtet. Als Messintervall wurden hier
$\SI{30}{\second}$ gewählt, daher wird von den Messdaten der Untergrund
\begin{equation}
	\langle N_U \rangle_{t=\SI{30}{\second}} = 
	\frac{\langle N_U \rangle_{t=\SI{30}{\second}}}{10} = 14 \pm 0,4
\end{equation}
abgezogen.

\begin{longtable}{c c c c}
	%\centering
	\caption{Messdaten zum Zerfall von Vanadium-52. Die Abweichung folgt aus der Poisson Verteilung
		mit $\Delta N = \sqrt{N}$. Für die korrigierten Zerfallsraten wurde der Untergrund 
	$14 \frac{\text{Zerfällen}}{\SI{30}{\second}}$ abgezogen.} \label{tab:vanadium} \\
		\hline
		$t / \si{\second}$ & $N$ & $\Delta N$ &$N$ Korrigiert \\
		%\midrule
		\hline
		\endhead
		\hline
		\endfoot
		30   	&      189  	& 14 &      175 \\
		  60   	&      197  	& 14 &      183 \\
		  90   	&      150  	& 12 &      136 \\
		 120   	&      159  	& 13 &      145 \\
		 150   	&      155  	& 12 &      141 \\
		 180   	&      132  	& 11 &      118 \\
		 210   	&      117  	& 11 &      103 \\
		 240   	&      107  	& 10 &       93 \\
		 270   	&       94  	& 10 &       80 \\
		 300   	&      100  	& 10 &       86 \\
		 330   	&       79  	&  9 &       65 \\
		 360   	&       69  	&  8 &       55 \\
		 390   	&       81  	&  9 &       67 \\
		 420   	&       46  	&  7 &       32 \\
		 450   	&       49  	&  7 &       35 \\
		 480   	&       61  	&  8 &       47 \\
		 510   	&       56  	&  7 &       42 \\
		 540   	&       40  	&  6 &       26 \\
		 570   	&       45  	&  7 &       31 \\
		 600   	&       32  	&  6 &       18 \\
		 630   	&       27  	&  5 &       13 \\
		 660   	&       43  	&  7 &       29 \\
		 690   	&       35  	&  6 &       21 \\
		 720   	&       19  	&  4 &        5 \\
		 750   	&       28  	&  5 &       14 \\
		 780   	&       27  	&  5 &       13 \\
		 810   	&       36  	&  6 &       22 \\
		 840   	&       25  	&  5 &       11 \\
		 870   	&       29  	&  5 &       15 \\
		 900   	&       18  	&  4 &        4 \\
		 930   	&       17  	&  4 &        3 \\
		 960   	&       24  	&  5 &       10 \\
		 990   	&       21  	&  5 &        7 \\
		1020   	&       25  	&  5 &       11 \\
		1050   	&       21  	&  5 &        7 \\
		1080   	&       24  	&  5 &       10 \\
		1110   	&       25  	&  5 &       11 \\
		1140   	&       17  	&  4 &        3 \\
		1170   	&       20  	&  4 &        6 \\
		1200   	&       19  	&  4 &        5 \\
		1230   	&       20  	&  4 &        6 \\
		1260   	&       18  	&  4 &        4 \\
		1290   	&       16  	&  4 &        2 \\
		1320   	&       17  	&  4 &        3 \\
\end{longtable}
\noindent
Für die im Zeitinterval $\Delta t$ gemessenen Zerfälle gilt
\begin{equation}
	\ln N_{\Delta t}(t) = \ln\left( N_0 \cdot \left(1 - e^{-\lambda \Delta t}\right) \right) - \lambda t.
	\label{eqn:logarithmisch}
\end{equation}
\noindent
In einer linearen Ausgleichsrechnung wurde die Funktion 
\begin{equation}
	f(x) = m \cdot x + b
	\label{eqn:mx+b}
\end{equation}
\noindent
an den Logarithmus der Messdaten angepasst. 
Dabei gilt zwischen \autoref{eqn:logarithmisch} und \autoref{eqn:mx+b} der 
Zusammenhang $m = -\lambda$.
\\
Die lineare Ausgleichsrechnung ergibt
\begin{equation}
	\lambda = (3,2 \pm 0,16) 10^{-3} \  \si{\second}^{-1}.
\end{equation}
\noindent Dann folgt die Halbwertszeit
\begin{equation}
	T = \SI{219\pm11}{\second}.
\end{equation}
Hier wird, wie bei der Untergrundrate, mit der gaußschen Fehlerfortplanzung gerechnet (Siehe 
\autoref{eqn:fehlerfortpflanzung}).
\\
In \autoref{fig:zerfallskurve} sind die Messdaten und die Ausgleichsfunktion zu sehen.

\begin{figure}[H]
	\centering
	\includegraphics{build/vanadium.pdf}
	\caption{Messdaten und Fit für die Zerfallskurve von Vanadium.}
	\label{fig:zerfallskurve}
\end{figure}


\subsection{Zerfallskurve und Halbwertszeit von Rhodium-104}
\label{sec:aus:rhodium}

Wie in \autoref{sec:Theorie} erklärt, besitzt Rhodium-104 zwei isomere Kerne mit unterschiedlichen 
Energien und somit zwei verschiedenen Halbwertszeiten.
\\
Die Messung der Zerfallsraten ergibt die Messwerte in \autoref{tab:rhodium}.
\begin{longtable}{c c c c}
	%\centering
	\caption{Messdaten zum Zerfall von Rhodium-104. Die Abweichung $\Delta N = \sqrt{N}$ folgt
		aus der Poisson Verteilung. Für die korrigierten Zerfallsraten wurde der Untergrund 
	$7 \frac{\text{Zerfällen}}{\SI{30}{\second}}$ abgezogen.} \label{tab:rhodium} \\
		\hline
		$t / \si{\second}$ & $N$ & $\Delta N$ &$N$ Korrigiert \\
		%\midrule
		\hline
		\endhead
		\hline
		\endfoot
		15  	& 667  	& 26         	& 660 \\
		 30  	& 585  	& 24         	& 578 \\
		 45  	& 474  	& 22         	& 467 \\
		 60  	& 399  	& 20         	& 392 \\
		 75  	& 304  	& 17         	& 297 \\
		 90  	& 253  	& 16         	& 246 \\
		105  	& 213  	& 15         	& 206 \\
		120  	& 173  	& 13         	& 166 \\
		135  	& 152  	& 12         	& 145 \\
		150  	& 126  	& 11         	& 119 \\
		165  	& 111  	& 11         	& 104 \\
		180  	&  92  	& 10         	&  85 \\
		195  	&  79  	&  9         	&  72 \\
		210  	&  74  	&  9         	&  67 \\
		225  	&  60  	&  8         	&  53 \\
		240  	&  52  	&  7         	&  45 \\
		255  	&  56  	&  7         	&  49 \\
		270  	&  53  	&  7         	&  46 \\
		285  	&  41  	&  6         	&  34 \\
		300  	&  36  	&  6         	&  29 \\
		315  	&  37  	&  6         	&  30 \\
		330  	&  32  	&  6         	&  25 \\
		345  	&  36  	&  6         	&  29 \\
		360  	&  38  	&  6         	&  31 \\
		375  	&  34  	&  6         	&  27 \\
		390  	&  40  	&  6         	&  33 \\
		405  	&  21  	&  5         	&  14 \\
		420  	&  35  	&  6         	&  28 \\
		435  	&  33  	&  6         	&  26 \\
		450  	&  36  	&  6         	&  29 \\
		465  	&  20  	&  4         	&  13 \\
		480  	&  24  	&  5         	&  17 \\
		495  	&  30  	&  5         	&  23 \\
		510  	&  30  	&  5         	&  23 \\
		525  	&  26  	&  5         	&  19 \\
		540  	&  28  	&  5         	&  21 \\
		555  	&  23  	&  5         	&  16 \\
		570  	&  20  	&  4         	&  13 \\
		585  	&  28  	&  5         	&  21 \\
		600  	&  17  	&  4         	&  10 \\
		615  	&  26  	&  5         	&  19 \\
		630  	&  19  	&  4         	&  12 \\
		645  	&  13  	&  4         	&   6 \\
		660  	&  17  	&  4         	&  10 \\
\end{longtable}
\noindent
Das Ziel dieser Messung ist es, beide Halbwertszeiten zu bestimmen. Dafür wird wie in \autoref{sec:aus:vanadium}
vorgegangen, wobei hier mit dem langsameren Zerfall angefangen wird.
\\
Zunächst wurden die Messdaten ab $\SI{300}{\second}$ betrachtet und die Funktion
\begin{equation}
	f(x) = -\lambda \cdot x + b
	\label{eqn:mx+b2}
\end{equation}
\noindent
an 
\begin{equation}
	\ln N_{\Delta t}(t) = \ln(N_0 \cdot (1 - e^{-\lambda \Delta t})) - \lambda t.
	\label{eqn:logarithmisch2}
\end{equation}
angepasst.
Die lineare Ausgleichsrechnung ergibt die Steigung
\begin{equation}
	\lambda_\text{l} = (3 \pm 0,6) 10^{-3}\ \si{\second}^{-1}.
\end{equation}
\noindent Dann folgt die Halbwertszeit des langsameren Zerfalls mit
\begin{equation}
	T_\text{l} = \SI{230\pm50}{\second}.
\end{equation}
\noindent
Als zweites soll die Halbwertszeit für den schnelleren Zerfall bestimmt werden. Dafür wird mit
$T_\text{l}$ berechnet, wie viele $\frac{\text{Impulse}}{\text{Messintervall}}$ am Anfang durch den schnellen Zerfall ausgelöst wurden.
Laut dem Zerfallsgesetz liegt das bei
\begin{equation}
	N_{\Delta t_\text{l}} (t) = N_{0_\text{l}} \left(1 - e^{-\lambda_\text{l}\Delta t}\right) 
	e^{-\lambda_\text{l}t }.
\end{equation}
\noindent
Wird $N_{\Delta t_\text{l}}(t)$ von den Messwerten bis $\SI{120}{\second}$ abgezogen, so liefert das
die Impulsrate des kurzlebigeren Atomkerns. Über den Logarithmus dieser Zerfallsrate, also
\begin{equation}
	\{ \ln(N_{\Delta t}(t_i) - N_{\Delta t_\text{l}} (t_i)) \},
\end{equation}
wird eine lineare Ausgleichsrechnung wie in \autoref{eqn:logarithmisch2} beschrieben durchgeführt. Diesmal
wird allerdings nur das Intervall von $t=0$ bis $t=\SI{120}{\second}$ betrachtet, da hier der schnelle 
Zerfall dominiert.
\\
Die Fit-Parameter liefern
\begin{equation}
	\lambda_\text{s} = (1,43 \pm 0,04) 10^{-2} \si{\second}^{-1}.
\end{equation}
Dann folgt unter Zuhilfenahme der Gaußschen Fehlerfortpflanzung
\begin{equation}
	T_\text{s} = (48\pm1,2)\si{\second}.
\end{equation}
\noindent
In \autoref{fig:rhodium} sind die Messdaten sowie die Ausgleichsgeraden skiziert.
\begin{figure}[H]
	\centering
	\includegraphics{build/rhodium.pdf}
	\caption{Messdaten und Fit für langsamen und schnellen Zerfall von Rhodium-104.}
	\label{fig:rhodium}
\end{figure}
