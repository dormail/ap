\section{Auswertung}
\label{sec:Auswertung}

\subsection{Bestimmung der Untergrundrate}
\label{sec:aus:untergrundrate}

Für eine exakte Auswertung ist die Bestimmung der Nullrate notwendig.
In sieben Messintervallen mit $t = \SI{300}{\second}$ wurden die 
Untergrundraten aus \autoref{tab:untergrundrate} aufgenommen.
\begin{table}
	\centering
	\caption{Messdaten zur Untergrundrate, Messintervall jeweils $t = \SI{300}{\second}$.}
	\label{tab:untergrundrate}
	\begin{tabular}{c c}
		\toprule
		$N_U$ & $\Delta N_U$ \\
		\midrule
		129 & 11.36 \\
		143 & 11.96 \\
		144 & 12 \\
		136 & 11.66 \\
		139 & 11.79 \\
		126 & 11.22 \\
		158 & 12.57 \\
		\bottomrule
	\end{tabular}
\end{table}

\noindent
Da die Zählraten Poisson verteilt sind, ist die Messunsicherheit durch $\Delta N = \sqrt{N}$ 
gegeben.
\\
Zu den sieben Messdaten lautet der Durchschnitt
\begin{equation}
	\langle N_U \rangle_{t = \SI{300}{\second}} = 139 \pm 4.
\end{equation}
Wobei dieser im folgenden auf die verschiedenen Messintervalle angepasst werden muss.
