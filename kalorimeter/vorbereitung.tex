\documentclass[a4paper]{article}

\usepackage[german]{babel}

\usepackage{biblatex}
\addbibresource{quelle.bib}

\usepackage{amsmath, amssymb}
\usepackage{siunitx}

\usepackage[margin=2.5cm]{geometry}

\begin{document}
\section{Wärmekapazitäten}
Zusammenhang zwischen molarer und spezifischer Wärmekapazität: 
$c = \frac{c_m}{M}$
\begin{enumerate}
	\item Wasser \cite{janaf}
		\begin{align*}
			c_m &= 75.349 \, \si{\joule\per\kelvin\per\mol}
			\quad
			M = 18.01528 \, \si{\gram\per\mol} 
			\\
			\Righarrow c &= 4.1825 \, \si{\joule\per\kelvin\per\kg}
		\end{align*}
	\item Aluminium \cite[16]{handbuch}
		\begin{align*}
			c_m &= 24.30 \, \si{\joule\per\kelvin\per\mol}
			\quad
			M = 26.981539 \, \si{\gram\per\mol} 
			\\
			\Rightarrow c &= 0.90062 \, \si{\joule\per\kelvin\per\kg}
		\end{align*}
	\item Kupfer \cite[133f]{handbuch}
		\begin{align*}
			c_m &= 24.44 \, \si{\joule\per\kelvin\per\mol}
			\quad
			M = 63.546 \, \si{\gram\per\mol} 
			\\
			\Rightarrow c &= 0.3846 \, \si{\joule\per\kelvin\per\kg}
		\end{align*}
	\item Glas \cite[944f]{handbuch} 
		\[
			c_p = 730 \,\si{\joule\per\kelvin\per\kg}
		\]
\end{enumerate}

\section{Gesetz von Dulong Petit}

\printbibliography

\end{document}
