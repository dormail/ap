\section*{Zielsetzung}
\label{sec:Zielsetzung}
In diesem Versuch soll mittels Ultraschallechographie die Schallgeschwindigkeit in Acryl bestimmt werden.

\section{Theorie}
\label{sec:Theorie}

\subsection{Grundlagen zu Schallwellen}
\label{sec:GrundlagenSchallwellen}

Schallwellen lassen sich gemäß ihrer Frequenz in verschiedene Bereiche aufteilen. Einige wichtige 
Frequenzbereiche sind
\begin{itemize}
	\item Infraschall ($f < \SI{16}{Hz}$),
	\item das vom Menschen wahrnehmbare Spektrum ($\SI{16}{Hz} < f < \SI{20}{kHz}$),
	\item Ultraschall ($\SI{20}{kHz} < f < \SI{1}{GHz}$),
	\item und Hyperschall ($f > \SI{1}{GHz}$).
\end{itemize}
In Luft breitet sich Schall als longitudinale Welle aus, die sich aufgrund von Druckschwankungen
\begin{equation}
	p(x,t) = p_0 + v_0 Z \cos(\omega t - kx)
\end{equation}
fortbewegt. 
Wie bei elektromagnetischen Wellen lassen sich auch bei Schallwellen Phänomene wie Reflexion und Brechung
beobachten. Im Gegensatz zum Licht ist bei Schall die Phasengeschwindigkeit druck-, dichte- und somit
materialabhängig. Für Flüssigkeiten wird das durch die Kompressibilität $\kappa$ ausgedrückt.
Damit lautet die Schallgeschwindigkeit in Flüssigkeiten
\begin{equation}
	\label{eqn:c_fl}
	c_\text{Fl} = \sqrt{\frac{1}{\kappa \rho}}.
\end{equation}
Die Betrachtung von Schallwellen in Festkörpern bringt jedoch mehrere Hürden mit sich: Aufgrund von
Schubspannungen sind hier auch transversale Wellen möglich. Diese haben dabei auch eine andere Schallgeschwindigkeit
als die longitudinalen Wellen im selben Festkörper. Für Festkörper lasst sich die Schallgeschwindigkeit
\begin{equation}
	\label{eqn:c_fe}
	c_{Fe} = \sqrt{\frac{E}{\rho}}
\end{equation}
berechnen. Dabei tritt der Elastizitätsmodul $E$ als Pendant zur Kompressibilität $\kappa^{-1}$ eine Flüssigkeit
auf. Die Schallgeschwindigkeit ist aber grundsätzlich richtungsabhängig.

\noindent
Schallausbreitung ist allermeistens verlustbehaftet. Die Intensität nimmt dabei exponentiell ab gemäß
\begin{equation}
	I(x) = I_0 \cdot e^{-\alpha}
\end{equation}
mit dem Absorptionskoeffizienten $\alpha$. Luft hat einen sehr großen Absorptionskoeffizienten, weswegen 
zwischen Schallgeber und zu untersuchendem Objekt ein Kontaktmittel verwendet wird.

\noindent
Wie oben schon beschrieben, kann bei Schall Reflexion beobachtet werden. Der Reflexionskoeffizient
\begin{equation}
	R = \left(\frac{Z_1 - Z_2}{Z_1 + Z_2} \right) 
	= \frac{\text{Intensität des reflekt. Schalls}}{\text{Intensität des einfallenden Schalls}}
\end{equation}
setzt sich aus den akkustischen Impedanzen 
\begin{equation}
	\label{eqn:impedanz}
	Z = \rho \cdot c
\end{equation}
der beiden Materialien zusammen.

\subsection{Erzeugung von Ultraschall}
\label{sec:Erzeugung von Ultraschall}

Ultraschall kann auf unterschiedlicher Weise erzeugt werden, hier wird die Erzeugung mittels
piezoelektrischer Kristalle erläutert. Eben erwähnter Kristall wird dazu in einem zeitlich
veränderlichen, elektrischen Feld zu Schwingungen angeregt. Bei diesen Schwingungen werden
Ultraschallwellen abgestrahlt, bei Resonanz (Anregefrequenz $\approx$ Eigenfrequenz) sind
sehr große Amplituden möglich.  Ein Vorteil vom Piezokristall ist, dass er auch als Schallempfänger
benutzt werden kann; die eintreffenden Schallwellen regen dann den Kristall zu Schwingungen an.

\subsection{Verwendung in der Medizin}
\label{sec:Verwendung in der Medizin}

Ein wichtiges Anwendungsgebiet des Ultraschalls liegt in der Medizin als bildgebendes Verfahren.
Zwei Verfahren sind das Durchschallungs- und das Impuls-Echo-Verfahren.

% Durchschallungs Verfahren
Beim \textit{Durchschallungs-Verfahren} wird mit getrenntem Sender und Empfänger gearbeitet, wobei der
Schallgang durch das Probenstück durchführt. Fehlstellen werden dann als abgeschwächtes Signal am
Empfänger deutlich, die Tiefe einer Fehlstell kann bei einer einzelnen Messung allerdings nicht
bestimmt werden.

% Impuls Echo
Beim \textit{Impuls-Echo-Verfahren} ist der Ultraschallsender gleichzeit auch der Empfänger. Das Verfahren
beruht auf der zuvor beschriebenen Reflexion an Grenzflächen, welche ebenfalls an der Fehlstelle in
der Probe auftritt. Nach ausgesendetem Impuls werden die reflektierten Schallsignale aufgenommen.
\\
Der Vorteil des Impuls-Echo-Verfahren ist der, dass auch die Tiefe der Fehlstelle detektiert
werden kann. Aus der Latenz des Echos kann bei bekannter Schallgeschwindigkeit die Tiefe gemäß
\begin{equation}
	\label{eqn:tiefe_fehlstelle}
	s = \frac12 ct
\end{equation}
bestimmt werden.

