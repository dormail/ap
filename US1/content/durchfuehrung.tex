\section{Durchführung}
\label{sec:Durchführung}

Nach dem theoretischen Hintergrund soll nun die Durchführung beschrieben werden. Von den erwähnten Verfahren wird
das Impuls-Echo-Verfahren angewendet, die zu untersuchenden Objekte sind ein Acrylquader und ein vergrößertes
Modell vom menschlichen Auge.

\subsection{Acrylquader mit Fehlstellen}
\label{sec:Acrylquader mit Fehlstellen}

Damit im Messprogramm die tatsächlichen Tiefen bekannt sind, sollen die im Acrylquader eingelassenen Fehlstellen
mit einer Schieblehre bemessen werden.
\\
Mit einer 2MHz-Sonde wird nun die Schalllaufzeit von den Rändern zu den Fehlstellen gemessen werden. Dabei muss
zwischen Sonde und Quader destiliertes Wasser oder Hydrogel aufgetragen werden. Für ein genaues Ergebniss sollen
möglichst viele Fehlstellen ausgemessen werden.
\\
Aus der Laufzeit kann mit den zuvor bestimmten, tatsächlichen Distanzen die Schallgeschwindigkeit bestimmt werden.

\subsection{Ausmessung des Augenmodells}
\label{sec:Ausmessung des Augenmodells}

Im zweiten Versuchsteil sollen die Ausmessungen im Inneren eines Augenmodells bestimmt werden. Mit leichtem Druck
wird die Sonde auf die Hornhaut gepresst und der Winkel variiert, bis das Echo von der Retina auf dem A-Scan zu
sehen ist. Mit bekannten Schallgeschwindigkeiten können die Distanzen einzeln berechnet werden.
\vspace{0.5cm}
\\
Beide Bauteile sollen nach dem Versuch gereinigt bzw. getrocknet werden.

