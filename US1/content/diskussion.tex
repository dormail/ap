\section{Diskussion}
\label{sec:Diskussion}
Für die Schallgeschwindigkeit c in Acrylglas wurde in der Auswertung des Versuchs ein Wert von $c=(2835.78 \pm 61.02) m/s$ ermittelt. Eine vorherige Literaturrecherche ergab einen Literaturwert von $c_{Lit}=2750 m/s$ für die Schallgeschwindigkeit in Acrylglas. Somit ist der im Versuch ermittelte Wert für c relativ nah am Literaturwert. Die geringe Abweichung vom Literaturwert ist mit systematischen Fehlern bei der Messung zu begründen. So führt unter anderem die Anpassungsschicht der Ultraschallsonden zu einem systematischen Fehler bei den Messungen. \newline
Bei der Untersuchung des Augenmodells hat sich als Gesamtlänge (bzw. Durchmesser) des Auges ein Wert von circa 19 $mm$ ergeben. Dies ist ein realistischer Wert. Geringfügige Abweichungen können damit erklärt werden, dass das Augenmodell schon etwas älter war und ein paar leichte Risse auf der Oberfläche aufwies, was den systematischen Fehler bei der Messung noch erhöhen kann.    
