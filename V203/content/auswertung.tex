\section{Auswertung}
\label{sec:Auswertung}
\subsection{Messung bis 1 Bar}
In \autoref{tab:bis1} sind die ermittelten Messwerte für den Messbereich bis 1 Bar aufgelistet.
\begin{table}[H]
  \centering
  \caption{Messwerte bis 1 \si{\bar}}
  \label{tab:bis1}
  \sisetup{table-format=2.1}
  \begin{tabular}{ccc}
  \toprule
  $T_{Wasser} \,/\, \si{\celsius}$ & $p \,/\, \si{\milli\bar}$ & $T_{Dampf}\,/\, \si{\celsius}$\\
  \midrule
  21  & 35  & 23   \\
  26  & 43  & 24   \\
  30  & 50  & 30   \\
  35  & 60  & 34.5 \\
  40  & 76  & 39   \\
  45  & 96  & 44.5 \\
  50  & 121 & 49   \\
  55  & 152 & 54   \\
  60  & 178 & 57.5 \\
  65  & 190 & 59   \\
  70  & 208 & 62   \\
  75  & 225 & 62   \\
  80  & 230 & 67   \\
  85  & 234 & 67   \\
  90  & 247 & 70   \\
  95  & 253 & 72   \\
  100 & 260 & 75   \\
  105 & 267 & 82   \\
  120 & 275 & 88   \\
  130 & 276 & 92   \\
  145 & 288 & 98  \\ 
  150 &400 &99\\
  152& 600 &100\\
  154 &800 &102\\
  155 &1000 &105 \\  
  \end{tabular} 
\end{table}
\noindent
Die Formel für den Zusammenhang des Logarithmus des Dampfdruckes und der reziproken absoluten Temperatur ist:
\begin{align}
  \ln(\frac{p}{p_0})=- \frac{L}{R} \frac{1}{T}+const 
\end{align}
\noindent
Bezogen auf diese Gleichung kann nun mit einem linearen Fit der Messwerte die Verdampfungswärme ermittelt werden.
In \autoref{fig:plot1} sind Messwerte und die entsprechende Ausgleichsgerade abgebildet. 
\begin{figure}[H]
  \centering
  \includegraphics[width=\textwidth]{build/plot1.pdf}
  \caption{Messwerte für den Messbereich bis 1\si[]{\bar} mit Ausgleichsgerade }
  \label{fig:plot1}
\end{figure}
\noindent
Die Ausgleichsrechnung ergibt für die Gerade
\begin{align}
  \ln(\frac{p}{p_0})&=- m \frac{1}{T}+b \\
  m&=(3647.434 \pm 223.452) \si{\kelvin}\\
  b&=223.452 \pm 0.664 .
\end{align}
Mit der Gaskonstanten $\symup{R}$ lässt sich nun aus der Steigung ein Wert für $L$ bestimmen:
\begin{align}
    L&=m  \symup{R} \\
    \leftrightarrow L&=\SI{30.326 \pm 1.858 }{\kilo\joule\per\mol} 
\end{align}\\
\noindent
Um die innere Verdampfungswärme $L_i$, pro Molekül zu bestimmen, wird die äußere Verdampfungswärme $L_a$ berechnet, um $L_i$ aus der Formel $L=L_i+L_a$ zu bestimmen.\\
Für die äußere Verdampfungswärme wird $T=373 \si{\kelvin}$, also $T\approx 100 \si{\celsius}$, angenommen, sodass sich $L_a$ über die Allgemeine Gasgleichung bestimmen lässt:
\begin{equation}
    L_a=pV=\symup{R} \cdot T = \SI{3101.2946}{\joule\per\mol} 
\end{equation}
Daraus folgt :
\begin{align}
    L_i&=L-L_a \\
    L_i&=\SI{27.225\pm 1.858 }{\kilo\joule\per\mol}
\end{align}
Für die Berechnung pro Molekül in $\si{\electronvolt}$ wird $L_i$ noch durch die Avogadro-Konstante $\symup{N_A}$ und die Elementarladung dividiert:
\begin{equation}
    L_i=\SI{0.2822 \pm 0.0193}{\electronvolt} 
\end{equation}

\subsection{Messung 1 bis 15 Bar}
Die gemessenen Werte für den Messbereich von 1 Bar bis 15 Bar sind in \autoref{tab:bis15hz} aufgelistet.
  \begin{table}[H]
    \centering
    \caption{Messwerte bis 15 Hz}
    \label{tab:bis15hz}
    \begin{tabular}{ll}
      \toprule
      $p \,/\, \si{\bar}$ & $T\,/\, \si{\celsius}$\\
      \midrule
    1  & 117   \\
    2  & 130   \\
    3  & 140   \\
    4  & 149   \\
    5  & 156   \\
    6  & 161   \\
    7  & 167   \\
    8  & 171   \\
    9  & 176   \\
    10 & 180   \\
    11 & 184   \\
    12 & 188   \\
    13 & 191   \\
    14 & 194.5 \\
    15 & 197  
    \end{tabular}
    \end{table}
\noindent
In diesem Teil wird zuerst die Clausius-Clapeyronsche Gleichung nach der Verdampfungswärme $L$ umgestellt:
    \begin{equation}
        L= T \cdot (V_D-V_F) \frac{\symup{d}p}{\symup{d}T}
        \label{eqn:clausL}
    \end{equation}
Da der Sättigunsdampfdruck hier nicht mehr vom Volumen abhängig ist, lässt sich $V_D$ nicht 
über die Allgemeine Gasgleichung bestimmen. Stattdessen wird folgende Näherung genutzt:
\begin{align}
  \symup{R}\cdot T &= \left( p + \frac{A}{V_D^2}\right)\cdot V_D  &
   &\text{mit} &
   A&=\SI{0.9}{\joule\cubic\metre\per\mol\squared} \\
   \implies V_D&=\frac{\symup{R}\cdot T}{2p} \pm \sqrt{\frac{\symup{R}^2\cdot T^2}{4p^2}-\frac{A}{p}} 
   \intertext{Da $V_F$ vernachlässigbar ist, kann L wie folgend berechnet werden:}
   L&=T\left(\frac{\symup{d}p}{\symup{d}t}\frac{\symup{R}\cdot T}{2p} \pm\sqrt{\frac{\symup{R}^2\cdot T^2}{4p^2}-\frac{A}{p}}\right)\frac{\symup{d}p}{\symup{d}t} \nonumber\\
   L&=\frac{T}{p}\left( \frac{\symup{R} \cdot T}{2} \pm\sqrt{\frac{\symup{R}^2\cdot T^2}{4}-Ap} \right)\frac{\symup{d}p}{\symup{d}t} \label{eqn:ohnedp}
   \intertext{Um $\frac{\symup{d}p}{\symup{d}t}$ zu nähern, wird eine Fit-Funktion 3.Grades
   für die Temperatur $T$ und den Druck $p$ bestimmt. Diese wird anschließend differenziert.}  \nonumber
\end{align}    

    Der Fit auf $p$ und seine dazugehörige Ableitung sind: 
    \begin{align*}
        p(T)&=a\cdot T^3+b\cdot T^2 + c\cdot T+d \\
        \frac{\symup{d}p}{\symup{d}t}&=3a\cdot T^2+2b\cdot T + c 
    \end{align*}
    Eine Ausgleichsrechnung in python mit der Funktion numpy.polyfit ergibt für die Parameter der Ausgleichskurve:
  
    \begin{align}
      a	&= (0.000007337 \pm 0.000001827) & \si{\bar\per\cubic\kelvin}\\
      b	&= (-0.007780 \pm 0.002363) & \si{\bar\per\kelvin\squared}\\
    c	&=(2.782  \pm 1.017 )&  \si{\bar\per\kelvin}\\
    d	&=(-336.017  \pm 145.664) & \si{\bar}
  \end{align}
  In \autoref{fig:plot2} sind die ermittelte Ausgleichskurve und die entsprechenden Messwerte dargestellt.
    \begin{figure}[H]
      \centering
      \includegraphics[width=\textwidth]{build/plot2.pdf}
      \caption{Messwerte für den Messbereich bis 15\si[]{\bar} mit Ausgleichskurve}
      \label{fig:plot2}
    \end{figure}
    \noindent
    Einsetzen der Fit-Funktionen in \autoref{eqn:ohnedp} führt zu:
    \begin{equation}
        L(T)=\left( \frac{\symup{R} \cdot T}{2} \pm\sqrt{\frac{\symup{R}^2\cdot T^2}{4}-Ap} \right)\frac{3a\cdot T^3+2b\cdot T^2 + c\cdot T}{a\cdot T^3+b\cdot T^2 + c\cdot T+d}
        \label{eqn:mitdp}
    \end{equation}