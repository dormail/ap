\section{Diskussion}
\label{sec:Diskussion}
Im ersten Teil der Auswertung wurden die Messwerte in einem Druckbereich bis 1 bar betrachtet. Als Wert für die Verdampfungswärme von Wasser wurde $L=\SI{30.326 \pm 1.858 }{\kilo\joule\per\mol}$  ermittelt. Im Vergleich mit dem Literaturwert $L_{Lit}=\SI{40.7}{\kilo\joule\per\mol}$\cite{verdampf} ist zu sehen dass der experimentell ermittelte Wert zwar in der richtigen Größenordnung liegt aber noch signikfikant abweicht. Die Abweichung beträgt $\symup{\Delta}L=\SI{10.374}{\kilo\joule\per\mol}$ bzw. $\symup{\Delta}L=25.5\%$. Dies kann daran liegen, dass die Evakuierung der Apparatur mit der Wasserstrahlpumpe nicht optimal war. Außerdem sind Ungenauigkeiten beim Ablesen der Messwerte nicht auszuschließen. \newline
Im Weiteren wurde aus den ermittelten Messwerten im Bereich von 1 bis 15 bar eine Funktion für die Verdampfungswärme genähert. Diese Funktion ist in \autoref{fig:plot3} dargestellt.
\begin{figure}[H]
    \centering
    \includegraphics[width=\textwidth]{build/plot3.pdf}
    \caption{Genäherte Funktion für die Verdampfungswärme L}
    \label{fig:plot3}
\end{figure}
\noindent
Aufgrund dieses Kurvenverlaufs kann man davon ausgehen, dass der Zusammenhang zwischen Verdampfungswärme L und Temperatur T hyperbolisch ist. Auffallend ist auch, dass der Zusammenhang für hohe Temperaturen nahezu in einen linearen Zusammenhang übergeht.