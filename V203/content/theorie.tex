\section*{Zielsetzung}
\label{sec:Zielsetzung}
In diesem Versuch soll der Übergang zwischen den Phasen flüssig und gasförmig quantitativ
untersucht werden. Ziel ist es, dabei die Dampfdruckkurve von Wasser im bereich von
$\SI{30}{\milli\bar}$ bis $\SI{15}{\bar}$ zu bestimmen.

\section{Theorie}
\label{sec:Theorie}
Der anfangs verwendete Begriff der ``Phase`` beschreibt hierbei einen räumlich
abgegrenzten Bereich, in dem sich ein Stoff in einem physikalisch homogenen Zustand
befindet. Anschauliche Beispiele sind die Aggregatszustände.
\\
Die Unterteilung dieser Phasen kann im Zustandsdiagramm stattfinden, wo der Druck $p$
gegen die Temperatur $T$ aufgetragen wird. Für Wasser lassen sich dann mit drei Kurven
die drei Aggregatszustände abgrenzen:
\begin{figure}[H]
	\centering
	\includegraphics[height=8cm]{images/Zustandsdiagramm.png}
	\caption{Zustandsdiagramm des Wassers (nur qualitativ) \cite{anleitung}}
	\label{fig:zustandsdiagramm}
\end{figure}
\noindent
Offensichtlich kann innerhalb eines Aggregatszustandes $p$ und $T$ frei gewählt werden
(solange die Kombination noch im entsprechenden Zustand liegt), das System hat also zwei Freiheitsgrade.
Sind Druck und Temperatur jedoch so, dass der Zustand nahe an einer der Kurven ist, ändert
sich das. Dort existieren zwei Phasen nebeneinander. Für die sog. \textbf{Dampfdruckkurve}, die
zwischen Tripelpunkt und kritischem Punkt verläuft und die Aggregatszustände flüssig und
gasförmig trennt, hat das System dann auch nur noch einen Freiheitsgrad.
\\
Der Verlauf folgt hauptsächlich aus der \textbf{Verdampfungswärme $L$}, welche eine
charakteristische Eigenschaft für einen Stoff ist. Problematisch ist, dass $L$ nicht
konstant ist, sondern selbst auch temperaturabhängig ist. Es existiert jedoch ein
beschränkter Bereicht, wo $L$ nahezu konstant ist. Dieser Bereich ist in diesem Versuch
von Interesse; dort soll die Dampfdruckkurve aufgenommen und $L$ bestimmt werden.


