\section*{Zielsetzung}
\label{sec:zielsetzung}

Schönes Michelson-Interferometer

\section{Theorie}
\label{sec:Theorie}
\subsection{Überlagerung von Lichtwellen, Begriff der Interferenz}
Elektromagnetische Wellen:
\begin{equation}
    \label{eqn:emwelle}
    \vec{E} = \vec{E_0} \cos(kx - \omega t - \delta) \; .
\end{equation}
Intensität:
\begin{equation}
    \label{eqn:Intensität}
    I = const |\vec{E}|
\end{equation}
Überlagerte Gesamtintensität:
\begin{equation}
    I_\text{ges} = 2 const \vec{E_0}^2 (1 + \symup{cos(\delta_2 - \delta_1))}.
\end{equation}

\cite{anleitung}
%\subsection{Kohärenz}
\subsection{Funktion Michelson-Interferometer, Brechungsindexaenderung?}
Brechungsindexänderung:
    \begin{equation}
        \label{eqn:brechungsindexunterschied}
        b \cdot \Delta n = z \frac{\lambda}{2}    
    \end{equation}
Brechungsindex:
    \begin{equation}
        \label{eqn:brechungsindex}
        n(p_0,T_0) = 1 + \symup{\Delta}n(p,p') \frac{T}{T_0} \frac{p_0}{p - p'}
    \end{equation}
        