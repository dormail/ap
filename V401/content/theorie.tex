\section*{Zielsetzung}
\label{sec:zielsetzung}

Im Versuch zum vorliegenden Protokoll wird mit einem Michelson-Interferometer x und y gemessen.

\section{Theorie}
\label{sec:Theorie}
In diesem Abschnitt werden die theoretischen Grundlagen des Versuchs erläutert. Zuerst werden die Phänomene der Interferenz und Kohärenz im Zusammenhang mit Lichtwellen erklärt. Anschließend 
\subsection{Überlagerung von Lichtwellen, Begriff der Interferenz}
Die elektrische Feldstärke einer elektromagnetischen Welle lässt sich mit ihrer Wellenzahl $k$, ihrer Kreisfrequenz $\omega$ und ihrem Phasenwinkel $\delta$ als
\begin{equation}
    \label{eqn:emwelle}
    \vec{E} = \vec{E_0} \cos(kx - \omega t - \delta) \; 
\end{equation}
darstellen. \newline
Die Intensität $I$ mehrerer überlagerter Lichtwellen, also der zeitliche Mittelwert der auf eine Flächeneinheit treffenden Lichtleistung, berechnet sich nach:
\begin{equation}
    \label{eqn:Intensität}
    I = const |\vec{E}|
\end{equation}
Überlagerte Gesamtintensität:
\begin{equation}
    I_\text{ges} = 2 const \vec{E_0}^2 (1 + \symup{cos(\delta_2 - \delta_1))}.
\end{equation}
Die Gesamtintensität kann auch, in Abhängigkeit der Phasenwinkel $\delta$ der beiden einzelnen Wellen, verschwinden. Dies ist der Fall für:
\begin{equation}
    \delta_2 - \delta_1 = (2n + 1)\pi \: , \; n = 0,1,2,3,... 
\end{equation}

\subsection{Kohärenz}
In der Regel treten allerdings keine Interferenzerscheinungen auf wenn sich die Wellen zweier unabhängiger Lichtquellen überlagern. Dies liegt daran, dass die Phasenwinkel $\delta_1$ und $\delta_2$ von 2 Lichtquellen in der Regel statistisch verteilte Funktionen der Zeit sind. Bei einer Mittelung über eine Zeit, die groß
gegen die Periodendauer $2\pi/\omega$ ist, wird der Interferenzterm daher zu 0.

\subsection{Funktion Michelson-Interferometer, Brechungsindexaenderung?}
Brechungsindexänderung:
    \begin{equation}
        \label{eqn:brechungsindexunterschied}
        b \cdot \Delta n = z \frac{\lambda}{2}    
    \end{equation}
Brechungsindex:
    \begin{equation}
        \label{eqn:brechungsindex}
        n(p_0,T_0) = 1 + \symup{\Delta}n(p,p') \frac{T}{T_0} \frac{p_0}{p - p'}
    \end{equation}
        