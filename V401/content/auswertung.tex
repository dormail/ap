\section{Auswertung}
\label{sec:Auswertung}

\subsection{Experimentelle Bestimmung der Wellenlänge des Laserlichts}
\label{sec:Experimentelle Bestimmung der Wellenlänge des Laserlichts}
Im ersten Versuchsteil sollte die Wellenlänge des verwendeten Lasers experimentell
bestimmt werden. Dazu wurde in acht Messungen der Motor um $\SI{5}{mm}$ in eine
Richtung verschoben. Mit der Hebeluntersetzung von 1:5,046 folgt die Verschiebung 
des Spiegels
\begin{equation}
	d = \SI{0,99}{mm}.
\end{equation}
Die automatische Zählung der Maxima ergab die Messwerte in \autoref{tab:wellenlaenge}.
% Messwerte erster Teil
\begin{table}
  \centering
  \caption{Messdaten zur Bestimmung der Wellenlänge des Lasers.}
  \label{tab:wellenlaenge}
  \sisetup{table-format=2.1}
  \begin{tabular}{c c}
  \toprule
  Messreihe & Anzahl Maxima\\
  \midrule
    1 & 2371 \\
    2 & 2365 \\
    3 & 2367 \\
    4 & 2383 \\
    5 & 2364 \\
    6 & 2361 \\
    7 & 2371 \\
    8 & 2375 \\
  \bottomrule
  \end{tabular}
\end{table}
Mit den Formeln für Mittelwert und Standardabweichung folgt die gemittelte Anzahl an
detektierten Maxima
\begin{equation}
	z = 2370+/-7.
\end{equation}
Mit der in \autoref{sec:Theorie} gezeigten Formel für die Wellenlänge folgt mit linearer
Fehlerfortpflanzung 
\begin{equation}
	\lambda = 2 \frac{d}{z} = (836 \pm 23) \, \si{nm}.
\end{equation}
Auf dem Etikett des Lasers war die Wellenlänge mit
\begin{equation}
	\label{eqn:wellenlaenge-real}
	\lambda_\text{real} = \SI{635}{\nano\meter}
\end{equation}
angegeben. Mit dem letzteren Wert wird im nächsten Abschnitt gerechnet.

\subsection{Messung des Brechungsindex in Luft}
\label{sec:Messung des Brechungsindex in Luft}
Im zweiten Teil soll nun der Brechungsindex von Luft bestimmt werden. Dazu wurde in
einem Lichtkanal eine Messzelle mit Länge $b = \SI{50}{mm}$ eingeschoben und der Gasdruck
wurde um $\Delta p = \SI{600}{torr} = \SI{7999,32}{Pa}$ herabgesetzt. Die automatische
Zählung ergab die Maximaanzahl in \autoref{tab:brechungsindex}.
% Messwerte
\begin{table}
  \centering
  \caption{Messdaten zur Bestimmung des Brechungsindex von Luft, aufgeschlüsselt nach
  Erhöhen bzw Verringerung des Drucks. In der vierten Messreihe war beim Ablassen des
  Unterdrucks kein Maximum messbar, daher wurde vor der Fünften der Laser neu justiert,
  genauso wie vor der Siebten.}
  \label{tab:brechungsindex}
  \sisetup{table-format=2.1}
  \begin{tabular}{c c c}
  \toprule
  Messreihe & Maxima Abpumpen & Maxima Vakuum ablassen\\
  \midrule
  1 & 22 & 28 \\
  2 & 22 & 30 \\
  3 & 21 & 22 \\
  4 & 18 & -  \\
  5 & 32 & 31 \\
  6 & 33 & 31 \\
  7 & 49 & 42 \\
  8 & 52 & 38 \\
  \bottomrule
  \end{tabular}
\end{table}
Analog zu \autoref{sec:Experimentelle Bestimmung der Wellenlänge des Laserlichts} folgt
dann die gemittelte Zahl der Maxima mit Standardabweichung
\[
	z = 38+/-8.
\]
Mit der gegebenen Wellenlänge (vgl. \autoref{eqn:wellenlaenge-real}) folgt dann
\begin{equation}
	\Delta n = \frac{z \lambda}{2b} = 0.00024 \pm 0.00005.
\end{equation}
Für Normalbedingungen folgt dann der Brechungsindex unter Normalbedingung
\begin{equation}
	n = 1 + \Delta n \frac{T}{T_0} \frac{p_0}{\Delta p} = 1.00033 \pm 0.00007
\end{equation}
Wie zuvor wurde mit Fehlerfortplanzung nach Gauß gerechnet. Als Temperatur im Labor 
wurden $\SI{20}{\celsius}$ und als Druck $p = p_0$ angenommen.
