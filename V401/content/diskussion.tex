\section{Diskussion}
\label{sec:Diskussion}

Für die Wellenlänge wurde der Wert
\[
	\lambda = (836 \pm 23) \, \si{nm}
\]
ermittelt. Dieser weist einen relativen Fehler $32\%$ auf im Vergleich zum gegebenen Wert. 
Für eine exaktere Bestimmung
von $\lambda$ hätten in den Messungen 700 Maxima mehr detektiert werden müssen. Eine
Fehlerquelle kann eine falsche Justierung des Lasers sein, wodurch der Detektor nicht
jedes Maxima als solches erkannt hat.
\\
Der Brechungsindex wurde mit 
\[
	n = 1.00033 \pm 0.00007 
\]
ziemlich nah am Literaturwert \cite{reflactiveindex} von
\[
	n_\text{lit} = 1.00027
\]
bestimmt. Die relative Abweichung von $\Delta n$ lag bei nur $11\%$, von $n$ selbst im
Nachkommastellenbereich. Letzteres ist der Tatsache geschuldet, dass $\Delta n \ll 1$ und 
$n \approx 1 + \Delta n$.
Auffällig ist der massive Anstieg der Messwerte nach einer neuen Justierung des Lasers,
was auch die zu niedrigen Werte im ersten Versuchsteil erklärt. Für den zweiten
Versuchsteil ist auch relevant, dass die Vakuumpumpe per Hand bedient wurde, was für eine
sehr ungleichmäßige Druckkurve gesorgt hat. Ein Resultat ist, dass die sehr schnell vorbei
ziehenden Maxima nicht vollständig detektiert werden konnten. Außerdem sorgte die
Handbedienung der Pumpe für leichte Erschütterungen am Versuchsaufbau.

