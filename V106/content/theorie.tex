\section*{Zielsetzung}
\label{sec:Zielsetzung}
In diesem Versuch sollen verschiedene Phänomene, die bei gekoppelten Pendeln auftreten,
untersucht werden. Im Fokus steht dabei die Schwebung.

\section{Theorie}
\label{sec:Theorie}
Wichtige Kenngrößen beim Pendel sind die Fadenlänge $l$ und Masse $m$. Wird das Pendel
ausgelenkt, so sorgt die Gewichtskraft mit ihrer tangentialen Komponente für ein
Drehmoment $M = D_p \cdot \phi$, wobei $\phi$ der Auslenkwinkel und $D_p$ die
Winkelrichtgröße des Pendels sind. Bei kleinen Winkeln ($\phi \leq \SI{10}{\deg}$) 
folgt aus der linearen Taylorentwicklung die Näherung $\sin(\phi) \approx \phi$.
Damit lässt sich die Bewegungsgleichung des Pendels
\begin{equation}
	J \cdot \ddot \phi + D_p \phi = 0
	\label{eqn:dgl-harm-osz}
\end{equation}
aufstellen, mit dem Trägheitsmoment $J$. Diese DGL wird durch eine harmonische Schwingung
mit der Frequenz
\begin{equation}
	\omega = \sqrt{\frac{D_p}{J}} = \sqrt{\frac{g}{l}}.
\end{equation}
Bemerkenswert ist dabei die Tatsache, dass $\omega$ komplett unabhängig von Pendelmasse
und dem Auslenkwinkel $\phi$ ist.
\\
In diesem Versuch sollen gezielt gekoppelte Pendel betrachtet werden; das wird hier durch
eine Feder realisiert. Diese sorgt für ein zusätzliches Drehmoment 
$M_1 = D_F (\phi_2 - \phi_1)$ aufs erste bzw. $M_2 = D_F (\phi_1 - \phi_2)$ auf das zweite
Pendel. Aus \hyperref{eqn:dgl-harm-osz} folgt ein System gekoppelter Differentialgleichung 
\begin{align}
	J \ddot \phi_1 + D \phi_1 &= D_F (\phi_2 - \phi_1), \\
	J \ddot \phi_2 + D \phi_2 &= D_F (\phi_1 - \phi_2).
\end{align}
Durch eine Variablensubstitution können diese entkoppelt werden und als zwei unabhängige,
harmonische Schwingungen dargestellt werden. Die Frequenzen dieser zwei Schwingungen
werden mit $\omega_1$ und $\omega_2$ bezeichnet, die Auslenkwinkel der Pendel $\alpha_i$.
Abhängig von den Anfangsbedingungen $\alpha(t = 0)$ und $\dot \alpha(t = 0)$ können dabei
drei Schwingungsarten auftreten.

\subsection{Auftretende Schwingungsarten}
\label{sec:Auftretende Schwingungsarten}
\begin{itemize}
	\item \textbf{Gleichsinnige Schwingung:} $\alpha_1 = \alpha_2$ \\
\end{itemize}

