\section{Diskussion}
\label{sec:Diskussion}
In diesem Abschnitt werden die Ergebnisse des Versuchs rekapituliert. Die prozentualen Abweichungen werden dabei nach 
\begin{equation*}
  \Delta x= | \frac{x_\text{exp} - x_\text{th}}{x_\text{th}} \cdot \SI{100}{\percent}|
\end{equation*}
berechnet. \newline
In \autoref{sec:gleichsinnig} wird die Schwingungsdauer für die gleichsinnige Schwingung einmal als Mittelwert der gemessenen Werte und einmal theoretisch über die Gravitationskonstante und die Pendellänge berechnet. Hier haben sich geringe bis praktisch keine Abweichungen zwischen den beiden Werten ergeben. Die Abweichungen betragen $\Delta T_{+,1}=14,1\%$ und $\Delta T_{+,2}=0\%$. \newline
Im Weiteren wurde in \autoref{sec:gegensinnig} die gegensinnige Schwingung untersucht. Auch hier wird die Schwingungsdauer der Schwingung einmal als Mittelwert der Messwerte und einmal theoretisch mit Hilfe der Kopplungskonstanten berechnet.
Dabei ergeben sich die Abweichungen $\Delta T_{-,1}=14,56\%$ und $\Delta T_{-,2}=6.94\%$. Auch für die gegensinnige Schwingung konnten also Werte mit geringen Abweichungen, die im Bereich üblicher Messungenauigkeiten liegen, ermittelt werden. \newline
Im letzten Abschnitt der Auswertung wurden Schwebungsperioden für die gekoppelte Schwingung ermittelt. Die Abweichungen der Messwerte von den berechneten Werten ist hier höher. So ergeben sich die Abweichungen $\Delta T_{S,1}=89,17\%$ und $\Delta T_{S,2}=86.71\%$. Die hohe Messungenauigkeit bei der Bestimmung der Schwebungsperioden kann teilweise damit erklärt werden, dass es deutlich schwieriger ist eine Schwebung, als eine gewöhnliche Schwingung, mit dem bloßen Auge genau zu erkennen. Dazu kommt, dass die Zeiten mit einem Handy gestoppt wurden, welches per Hand bedient wurde, was zu recht hohen Messungenauigkeiten führen kann, da die Stoppuhr auf dem Handy und das Pendel gleichzeitig beobachtet werden müssen. Dazu kommt, dass bei der Bestimmung der Schwebungsperioden jeweils nur eine Periode gemessen wurde, was ungenauer ist als beispielsweise 3 Perioden, wie bei den Schwingungsdauern, zu messen. 