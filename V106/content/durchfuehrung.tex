\section{Durchführung}
\label{sec:Durchführung}
Im Folgenden soll auf die Durchführung der Messung eingegangen werden. Die gesamte
Messreihe wird dabei mit zwei verschiedenen Pendellängen durchgeführt. In beiden sollen
folgende Werte bestimmt werden:
\begin{itemize}
	\item Die Schwingungsdauern der einzelnen Pendel frei schwingend $T_i$,
	\item Die Schwingungsdauer $T_+$ der gleichphasigen Schwingung,
	\item $T_-$ für gegenphasige Schwingung,
	\item Schwingungsdauer $T$ und Schwebungsdauer $T_S$ für die gekoppelte
		Schwingung.
\end{itemize}
Damit erfolgt die Messung in vier Schritten
\begin{enumerate}
	\item \textbf{Messung der Schwingungsdauern $T_i$ und $T_+$:} \\
		Zunächst wird die Feder entnommen und die Pendel werden jeweils gemäß
		Kleinwinkelnäherung um $\alpha < \SI{10}{\deg}$ ausgelenkt. Für jedes
		Pendel werden zweimal zehn Messwerte zur Periodendauer der Schwingung
		(von jedem Experimentierenden je zehn) aufgenommen. \\
		Da gemäß \autoref{eqn:T_+} $T_+ = T_i$ gilt, kann die Messung des Falls
		gleichsinniger Schwingung ausgelassen werden. \\
		Da eine Periodendauer sehr kurz sein kann, bietet es sich an, mehrere
		Perioden aufzuzeichnen und die Zeit entsprechend zu dividieren.
		\label{sec:durch-mess1}
	\item \textbf{Messung von $T_-$:} \\
		Die Feder wird zwischen die Pendel gespannt, die Höhe der Feder ist dabei
		egal. Danach werden die Pendel so ausgelenkt, dass die Winkel 
		entgegengesetzt zueinander sind ($\alpha_1 = -\alpha_2$). Dabei muss
		dadrauf geachtet werden, dass die Pendel sich in der Mitte nicht treffen.
		Auch hier werden von beiden Experimentierenden je zehn Messwerte der
		Periodendauer aufgenommen. Auch hier bietet sich der Trick, mehrere
		Intervalle aufzunehmen, an.
	\item \textbf{Messung der Schwebungssdauer $T_S$:} \\
		Für diese Messung muss ein Pendel ausgelenkt werden, während sich das
		Andere in Ruhelage befinden soll. Als Anfang und Ende einer Periode 
		kann hier der Stillstand eines Pendels gewaḧlt werden. Die Messung erfolgt
		in Analogie zu der zuvor beschriebenen. Hier genügen jedoch fünf Messwerte
		pro Studierender.
	\item \textbf{Messung der Schwingungsdauer beim Fall gekoppelter Schwingung $T$:} \\
		Wie in \autoref{sec:durch-mess1} ohne Kopplung, wird hier jetzt die
		Schwingung eines Pendels bei einer gekoppelten Schwingung betrachtet.
		Dabei muss innerhalb eines Schwebungsintervalls die Periodendauer gemessen
		werden. Wenn möglich, kann auch hier mit mehreren Intervallen gearbeitet
		werden. Wie in den ersten beiden Messungen sollen hier auch zweimal zehn
		Messwerte aufgenommen werden.
\end{enumerate}

