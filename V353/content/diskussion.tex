\section{Diskussion}
\label{sec:Diskussion}
Durch die drei Messreihen konnten für die Zeitkonstante des betrachteten Systems drei
Werte für die Zeitkonstante gewonnen werden:

\begin{align}
	RC_1 &= (100.9 \pm 6.5) \si{\micro s} &
	\label{eqn:RC_1}
	     &\text{ Bestimmung durchs Entladeverhalten}
	\\
	RC_2 &= (1076.7 \pm 6.8) \si{\micro s} &
	\label{eqn:RC_2}
	     &\text{ Bestimmung durchs Amplitudenverhältnis}
	\\
	RC_3 &= (891.786 \pm 0.004) \si{\micro s} &
	\label{eqn:RC_3}
	     &\text{ Bestimmung durch Phasenverschiebung}
\end{align}

Dabei ist auffällig, dass der erste Wert von den anderen um eine ganze Größenordnung
abweicht, während die letzteren nur um $20.7\%$ voneinander abweichen. Dies kann, weil die 
Abweichung ziemlich genau der Faktor 10 ist, an einem Bedienfehler, wie einer falsch eingestellten
Zeitskala am Oszilloskop, liegen.

Allerdings ist auch die Differenz zwischen $RC_2$ und $RC_3$ signifikant, diese lässt sich jedoch
durch systematische Fehler erklären. Für eine exakte Bestimmung der Phasenverschiebung müssen
die Y-Achsen von $U_C$ und $U_G$ am Oszilliskop geeicht werden, sodass die Signale symmetrisch
zur X-Achse liegen. Dies war für $U_G$ nicht möglich, da der Drehregler für
die Y-Verschiebung des zweiten Inputs am Oszilloskop abgebrochen war.

Dazu flackerte der Bildschirm des Oszilloskops nach einigen Sekunden, sodass die Werte nie lange 
deutlich ablesbar waren, vor allem die erste Messung wurde dadurch erheblich erschwert, 
was auch die Abweichung von $RC_1$ erklärt.
