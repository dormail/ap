\section{Auswertung}
\label{sec:Auswertung}

\subsection{Entladeverhalten des RC-Kreises}
\label{sec:4a-auswertung}

In der ersten Messreihe wurde die Entladung des RC-Kreises untersucht. Dabei ergaben sich
die Messwerte in \autoref{tab:messdaten-4a}.

\begin{table}
\centering
\caption{Messergebnisse zum Entladevorgang der RC-Schaltung}
\label{tab:messdaten-4a}
\sisetup{table-format=2.1}
\begin{tabular}{c c}
\toprule
t / (\si{\micro s}) & U / \si{\volt} \\
\midrule
  0  	&2.10 \\
 10  	&1.90 \\
 20  	&1.80 \\
 30  	&1.70 \\
 40  	&1.55 \\
 50  	&1.40 \\
 60  	&1.30 \\
 70  	&1.20 \\
 80  	&1.10 \\
 90  	&0.90 \\
100  	&0.80 \\
120  	&0.65 \\
150  	&0.30 \\
160  	&0.25 \\
\bottomrule
\end{tabular}
\end{table}

Wie in \autoref{sec:Theorie} gezeigt, wird dieses Verhalten durch die Funktion
\begin{equation}
	U_C(t) = U_C(0) \cdot \exp\left(-\frac{t}{RC}\right)
\end{equation}

Mit einer Ausgleichsrechnung wurde der Wert
\begin{equation}
	RC = (100.9 \pm 6.5) \si{\micro s}
	\label{eqn:ergebnis-4a}
\end{equation}
ermittelt. In \autoref{fig:plot-4a} sind die Messdaten und die Theoriekurve zum ermittelten Wert zu sehen.

\begin{figure}
	\centering
	\includegraphics{build/4a.pdf}
	\caption{Messdaten und Theoriekurve}
	\label{fig:plot-4a}
\end{figure}

\subsection{Amplitudenverhältnis des RC-Kreis}
\label{sec:4b-auswertung}

\subsection{Phasenverschiebung des RC-Kreis}
\label{sec:4c-auswertung}

\subsection{Integrator-Funktion des RC-Kreises}
\label{sec:4d-auswertung}

