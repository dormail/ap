\section{Auswertung}
\label{sec:Auswertung}

\subsection{Emissionsspektrum einer Cu-Röntgenröhre}
\label{sec:Emissionsspektrum einer Cu-Röntgenröhre}
Als erstes wurde das Emissionsspektrum der Cu-Röntgenröhre aufgezeichnet, um es hier
auswerten zu können. Als Messwerte wurden die vorgegebenen Messwerte verwendet. Diese sind
in \autoref{fig:emiss} grafisch dargestellt.
\begin{figure}
	\centering
	\includegraphics{build/EmisionCu.pdf}
	\caption{Emissionsspektrum der Cu-Röntgenröhre.}
	\label{fig:emiss}
\end{figure}
Für die $K_i$-Kanten wurde die Scipy Funktion zur peak-Findung verwendet. Diese ergab die
Winkel
\begin{equation}
	\theta_{K_\alpha} = (22,5 \pm 0,25)^\circ
	\qquad
	\theta_{K_\beta} = (20,2 \pm 0,25)^\circ.
\end{equation}
Da dies in erster Ordnung ist folgt die Wellenlänge des Lichts mit
\begin{equation}
	\lambda = 2 d \sin\theta.
\end{equation}
Die zugehörige Energie der Kanten folgen dann mit gausscher Fehlerfortplanzung aus dem
Zusammenhang $E = hc / \lambda$. Für die oben ermittelten Winkel folgen die Energien
\begin{equation}
	E_{K_\alpha} = (8040 \pm 80) \si{eV}
	\qquad
	E_{K_\beta} = (8910 \pm 110) \si{eV}.
\end{equation}

\subsection{Transmission}
\label{sec:Transmission}
Auch in diesem Teil wurde mit vorgegebenen Messwerten gearbeitet. Diese sind in
\autoref{tab:trans} angegeben.
\begin{table}
    \centering
    \begin{tabular}{c c c}
        \toprule
        $\alpha$ [°] & $N_0$ [Imp/s] & $N_\text{Al}$ [Imp/s] \\
        \midrule
        7.0     &  226.0 & 113.5 \\
        7.1     &  232.0 & 112.0 \\
        7.2     &  240.5 & 112.0 \\
        7.3     &  248.0 & 113.5 \\
        7.4     &  255.0 & 115.0 \\
        7.5     &  262.0 & 113.5 \\
        7.6     &  269.0 & 113.0 \\
        7.7     &  276.0 & 114.5 \\
        7.8     &  281.0 & 114.0 \\
        7.9     &  289.5 & 112.0 \\
        8.0     &  295.0 & 109.5 \\
        8.1     &  300.0 & 109.0 \\
        8.2     &  308.5 & 108.0 \\
        8.3     &  311.0 & 106.0 \\
        8.4     &  317.0 & 104.5 \\
        8.5     &  324.0 & 101.5 \\
        8.6     &  328.5 & 100.0 \\
        8.7     &  332.5 & 100.5  \\
        8.8     &  337.0 & 97.5  \\
        8.9     &  340.5 & 95.0  \\
        9.0     &  348.0 & 92.5  \\
        9.1     &  350.0 & 89.5  \\
        9.2     &  353.0 & 88.0  \\
        9.3     &  356.5 & 84.5 \\
        9.4     &  359.0 & 83.0 \\
        9.5     &  363.5 & 81.0 \\
        9.6     &  367.0 & 78.5 \\
        9.7     &  369.0 & 76.0 \\
        9.8     &  370.5 & 74.0 \\
        9.9     &  375.0 & 72.0 \\
        10.0&  375.5 & 68.5 \\
        \bottomrule
    \end{tabular}
    \caption{Gegebene Messwerte zur Transmission.}
    \label{tab:trans}
\end{table}
Zur Bereinigung der Daten muss die Totzeitkorrektur 
\[
	I_i = \frac{N_i}{1 - \tau N_i}
\]
für das Geiger-Müller-Zählrohr angewendet werden.
Die Messdaten sind in \autoref{fig:trans} grafisch dargestellt.
\begin{figure}
	\centering
	\includegraphics{build/transmission.pdf}
	\caption{Emissionsspektrum der Cu-Röntgenröhre.}
	\label{fig:trans}
\end{figure}
Für den nachfolgenden Teil wird eine Ausgleichgerade der Form 
\[
	T(x) = a\cdot x + b
\]
benötigt. Der Scipy curve\_fit ergab hier die Fitparameter
\begin{equation}
	a = (-1,518 \pm 0,024) \cdot 10^{10}
	\qquad
	b = (1,225 \pm 0,014).
\end{equation}

\newpage
\subsection{Compton-Wellenlänge}
\label{sec:Compton-Wellenlänge}
Für die Bestimmung der Compton-Wellenlänge müssen drei weitere Werte gemessen werden. Hier
wurden folgende Werte aufgezeichnet:
\begin{align*}
	&\text{ohne Al-Absorber} & & I_0 = 2731 \, \text{Impulse} \\
	&\text{mit Al-Absorber zw. Röntgenröhre und Streuer} & & I_0 = 2731 \, \text{Impulse} \\
	&\text{ohne Al-Absorber zw. Streuer und GM-Rohr} & & I_0 = 2731 \, \text{Impulse}
\end{align*}
Daraus lassen sich die Transmissionen
\begin{equation}
	T_1 = 0,432 \quad
	T_2 = 0,375
\end{equation}
berechnen. Die Ausgleichsgerade aus dem vorherigen Abschnittlässt sich umformen zu
\[ \alpha = \frac{T - b}{a}. \]
Mit der Braggschen Bedingung folgt 
\[
	\lambda = 2d \sin\alpha.
\]
Mit den Fitparametern und den errechneten Transmissionen folgen die Wellenlängen
\begin{equation}
	\lambda_1 = (5,219 \pm 0,12) \cdot 10^{-11} \si{\meter}
	\quad
	\lambda_2 = (5,595 \pm 0,13) \cdot 10^{-11} \si{\meter}.
\end{equation}
Die Comptonwellenlänge ist dann die Differenz
\[
	\Delta \lambda = (3,76 \pm 0,06) \cdot 10^{-11} \si{\meter}.
\]
	
\end{document}
