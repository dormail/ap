\section{Diskussion}
\label{sec:Diskussion}
Die wesentlichen Ergebnisse sind in \autoref{tab:mess} zusammengefasst.
\begin{table}
	\centering
	\begin{tabular}{c c @{${}\pm{}$} c c}
		\toprule
	& \multicolumn{2}{c}{Messwert} 
	& Literaturwert \\
	\midrule
		$E_{K_\alpha} [\text{eV}]$ & 8040 & 80 & 8060\\
		$E_{K_\beta} [\text{eV}]$ & 8910 & 110 & 8920\\
		$\lambda_{C} [\text{pm}]$ & 3,76 & 0,06 & 2,43\\
		\bottomrule
	\end{tabular}
	\caption{Ergebnisse von Messung bzw. Auswertung und zugehörige Literaturwerte
	\cite{ld-didactic.de} \cite{nist}.}
	\label{tab:mess}
\end{table}
Die relative Abweichung für die Energiewerte liegt bei $0,25\%$ für $K_\alpha$ 
bzw. $0,11\%$ für $K_\beta$. Das deutet auf eine sehr gute Messung hin. Die Abweichung der
Compton-Wellenlänge liegt jedoch bei $54,73\%$. Da hier aber eine sehr kleine Größe
bestimmt werden sollte, ist ein kleiner relativer Fehler sehr schwierig.
\\
Da der Versuch online stattfand kann zu Fehlern bei der Durchführung nichts gesagt werden.

\section{Compton-Effekt bei sichtbarem Licht}
\label{sec:Compton-Effekt bei sichtbarem Licht}
Optisches Licht hat Wellenlänge im Bereich der Nanometer, welche mehrere Größenordnungen
über dem Picometer liegt, der Größenordnung der Comptonwellenlänge. Daher ist kein
Compton-Effekt bei sichtbarem Licht zu erwarten.

