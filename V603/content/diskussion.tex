\section{Diskussion}
\label{sec:Diskussion}
Die wesentlichen Ergebnisse sind in \autoref{tab:mess} zusammengefasst.
\begin{table}
	\centering
	\begin{tabular}{c c @{${}\pm{}$} c c}
		\toprule
	& \multicolumn{2}{c}{Messwert} 
	& Literaturwert \\
	\midrule
		$E_{K_\alpha} [\text{eV}]$ & 8040 & 80 & 8060\\
		$E_{K_\beta} [\text{eV}]$ & 8910 & 110 & 8920\\
		$\lambda_{C} [\text{pm}]$ & 3,76 & 0,06 & 2,43\\
		\bottomrule
	\end{tabular}
	\caption{Ergebnisse von Messung bzw. Auswertung und zugehörige Literaturwerte
	\cite{ld-didactic.de} \cite{nist}.}
	\label{tab:mess}
\end{table}
Die relative Abweichung für die Energiewerte liegt bei $0,25\%$ für $K_\alpha$ 
bzw. $0,11\%$ für $K_\beta$. Das deutet auf eine sehr gute Messung hin. Die Abweichung der
Compton-Wellenlänge liegt jedoch bei $54,73\%$. Da hier aber eine sehr kleine Größe
bestimmt werden sollte, ist ein kleiner relativer Fehler sehr schwierig.
\\
Eine Fehlerquelle steckt hier in der Verwendung des Geiger-Müller-Zählrohrs. Es
wurde keine Untergrundrate gemessen oder angegeben, wodurch die Werte einen gewissen
Fehler nach oben aufweisen können. Ferner ist die Berechnung der Comptonwellenlänge
numerisch instabil, so können kleine Fehler in den ursprünglichen Messwerten in einem
großen Fehler in $\lambda_C$ resultieren. Das ist der Tatsache geschuldet, dass sie die
Differenz zweier Größen ist, welche um etwa eine Größenordnung größer sind. Es wurde auch
nur eine Messreihe gemacht, durch mehrfache Wiederholung wäre eine genauere Bestimmung von
$\lambda_C$ möglich.
\\
Anhand dem Ergebnis für die $K$-Kanten wird aber deutlich, dass die Messung insgesamt gut
ablief, was der Durchführung durch einen Computer geschuldet sein dürfte.

\section{Compton-Effekt bei sichtbarem Licht}
\label{sec:Compton-Effekt bei sichtbarem Licht}
Optisches Licht hat Wellenlänge im Bereich der Nanometer, welche mehrere Größenordnungen
über dem Picometer liegt, der Größenordnung der Comptonwellenlänge. Daher ist kein
Compton-Effekt bei sichtbarem Licht zu erwarten.

