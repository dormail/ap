\section{Durchführung}
\label{sec:Durchführung}
Im nächsten Abschnitt wird auf die Durchführung eingegangen. Abgesehen von der Messung des
Klirrfaktors beruhen die Messungen auf der 
Nullmethode, die Brückenspannung wird dabei mit einem digitalen Oszillograph gemessen.

\subsection{Wheatstonesche Brücke}
\label{sec:ExecWheatstonescheBrücke}

Hier sollen zwei unbekannte Widerstände bestimmt werden. Da \autoref{eqn:wheatstone-rx} nicht von 
$R_3$ und $R_4$, sondern nur von deren Verhältnis abhängt,
werden diese hier mit durch ein Potentiometer ersetzt.

Es wird das Verhältnis $R_3/R_4$ variiert, bis mit dem Oszillographen ein Minimum der Brückenspannung
gefunden wurde. Die Speisespannung wird auf $10\si{\volt}$ gestellt. Zur Fehlerbestimmung wird $R_2$
(siehe \autoref{fig:wheatstone-bruecke}) variiert.
