\section{Durchführung}
\label{sec:Durchführung}
Im nächsten Abschnitt wird auf die Durchführung eingegangen. Abgesehen von der Messung des
Klirrfaktors beruhen die Messungen auf der 
Nullmethode, die Brückenspannung wird dabei mit einem digitalen Oszillograph gemessen.

\subsection{Wheatstonesche Brücke}
\label{sec:exec-wheatstone}

Hier sollen zwei unbekannte Widerstände bestimmt werden. Da \autoref{eqn:wheatstone-rx} nicht von 
$R_3$ und $R_4$, sondern nur von deren Verhältnis abhängt,
werden diese hier mit durch ein Potentiometer ersetzt.

Es wird das Verhältnis $R_3/R_4$ variiert, bis mit dem Oszillographen ein Minimum der Brückenspannung
gefunden wurde. Die Speisespannung wird auf $10\si{\volt}$ gestellt. Zur Fehlerbestimmung wird $R_2$
(siehe \autoref{fig:wheatstone-bruecke}) variiert.

\subsection{Kapazitätsmessbrücke}
\label{sec:exec-kapazitaetsmessbrueck}

Die Durchführung bei der Kapazitätsmessbrücke erfolgt analog zu der bei der Wheatstoneschen Brücke.

Der Hauptunterschied liegt dadrin, dass nun zwei veränderliche Bauteile im Aufbau verbaut sind. Für die
Nullmethode wird an einem Bauteil der Wert verändert, bis das Oszilloskop ein Minimum in der Brückenspannung
verzeichnet. Danach wird dann am zweiten variablen Bauteil ein Minimum gesucht.
\\
Dies wird so lange durchgeführt, bis man ein totales Minimum gefunden hat, d.h. dass eine Veränderung eines
Wertes die Brückenspannung erhöht.

Da für die Bestimmung eines Wertes $R_2$ variert wird, muss für die Fehlerbestimmung $C_2$ in den 
verschiedenen Messreihen variert werden.

\subsection{Induktivitätsmessbrücke}
\label{sec:exec-induktivitätsmessbrücke}

Die Durchführung findet komplett analog zu der in \autoref{sec:exec-kapazitaetsmessbrueck} statt; zur
Fehlerbestimmung wird $L_2$ variert.

\subsection{Maxwell-Brücke}
\label{sec:exec-maxwell-bruecke}

Der Aufbau folgt gemäß \autoref{fig:maxwell-schaltplan}, für die Nullmethode werden $R_3$ und 
$R_4$ durch Potentiometer ersetz welche so lange variiert werden, bis die Brückenspannung 
minimal ist.

Es werden die gleichen unbekannten Induktivitäten wie bei der Induktivitätsmessbrücke 
eingesetzt.

\subsection{Wien-Robinson-Brücke und Klirrfaktor}
\label{sec:exec-wien-robinson-bruecke}

Der Aufbau erfolgt gemäß dem Schaltplan in \autoref{fig:wien-robinson-schaltplan}. Im 
Frequenzbereich von $20\si{\hertz}$ bis $30\si{\kilo\hertz}$ wird die Brückenspannung gemessen. 
Dabei soll auch die Frequenz $\omega_0$ gefunden werden, bei der die Brückenspannung minimal ist.

Für die Berechnung des Klirrfaktors muss auch die Brückenspannung im Minimum $\omega_0$ notiert werden.
