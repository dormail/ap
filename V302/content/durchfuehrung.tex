\section{Durchführung}
\label{sec:Durchführung}
Im nächsten Abschnitt wird auf die Durchführung eingegangen. Abgesehen von der Messung des
Klirrfaktors beruhen die Messungen auf der 
Nullmethode, die Brückenspannung wird dabei mit einem digitalen Oszillograph gemessen.

\subsection{Wheatstonesche Brücke}
\label{sec:exec-wheatstone}

Hier sollen zwei unbekannte Widerstände bestimmt werden. Da \autoref{eqn:wheatstone-rx} nicht von 
$R_3$ und $R_4$, sondern nur von deren Verhältnis abhängt,
werden diese hier mit durch ein Potentiometer ersetzt.

Es wird das Verhältnis $R_3/R_4$ variiert, bis mit dem Oszillographen ein Minimum der Brückenspannung
gefunden wurde. Die Speisespannung wird auf $10\si{\volt}$ gestellt. Zur Fehlerbestimmung wird $R_2$
(siehe \autoref{fig:wheatstone-bruecke}) variiert.

\subsection{Kapazitätsmessbrücke}
\label{sec:exec-kapazitaetsmessbrueck}

Die Durchführung bei der Kapazitätsmessbrücke erfolgt analog zu der bei der Wheatstoneschen Brücke.

Der Hauptunterschied liegt dadrin, dass nun zwei veränderliche Bauteile im Aufbau verbaut sind. Für die
Nullmethode wird an einem Bauteil der Wert verändert, bis das Oszilloskop ein Minimum in der Brückenspannung
verzeichnet. Danach wird dann am zweiten variablen Bauteil ein Minimum gesucht.
\\
Dies wird so lange durchgeführt, bis man ein totales Minimum gefunden hat, d.h. dass eine Veränderung eines
Wertes die Brückenspannung erhöht.

\subsection{Induktivitätsmessbrücke}
\label{sec:exec-induktivitätsmessbrücke}

\subsection{Maxwell-Brücke}
\label{sec:exec-maxwell-bruecke}

\subsection{Wien-Robinson-Brücke}
\label{sec:exec-wien-robinson-bruecke}

\subsection{Bestimmung des Klirrfaktors}
\label{sec:exec-klirrfaktors}
