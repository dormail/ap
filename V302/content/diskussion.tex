\section{Diskussion}
\label{sec:Diskussion}

Die Ergebnisse für Kapazität 9 sind schlecht ausgefallen (siehe \autoref{sec:akap}). So weisen die Werte einen hohen Fehler auf und der Verlustwiderstand $R_x$ fällt sehr hoch aus. Dies ist mit Problemen bei der Messung zu erklären. So wurde bei der Bestimmung der Daten zu Wert 9 kein Spannungsminimum gefunden, sondern der Rand des Voltmeters erreicht bevor die Spannung bei einem Minimum angekommen war.\newline
Im Gegensatz zu den Ergebnissen für Wert 9 wirken die Werte für die Induktivität (Wert 19) gut. So liegen die ermittelten Werte per Maxwell-Brücke im Fehlerbereich der berechneten Werte mit der Induktivitätsmessbrücke.\newline
In \autoref{fig:plot} ist zu sehen, dass die Kurve der Messdaten zur Wien-Robinson-Brücke nahe an der Theorie-Kurve liegen. Bei der Messung wurde darauf geachtet besonders viele Messwerte in der Nähe des Minimums aufzunehmen. Dies führte dazu, dass die Kurve der Messdaten in der Nähe des Minimums besonders gut mit der Theorie-Kurve übereinstimmt. 