\section{Auswertung}
\label{sec:Auswertung}

\subsection{Wheatstone-Brücke}
Für die Messung des Widerstandes (Wert 11) wurden drei Messreihen vorgenommen, bei denen $R_2$ variiert wurde und das Potentiometer($R_3, R_4=1\si{\kilo}\si{\ohm}-R_3$) abgestimmt wurde, sodass ein Spannungsminimum entsteht. Die eingesetzten, gemessenen und berechneten  Widerstände wurden in \autoref{tab:WS} aufgelistet. Der Widerstand $R_x$ kann mit diesen Werten nach \autoref{eqn:wheatstone-rx} berechnet werden.

Mit der für das Potentiometer (also für $R_3/R_4$) gegebenen Messungenauigkeit von $0.5\%$ und der Standardabweichung für $R_x$ von $\sigma_{R_x}=1.921 \si{\ohm}$ lässt sich auch ein Fehler bestimmen. Für $R_x$ ergibt sich dann: 

für $R_x$:
\begin{align}
  R_x=(490.9 \pm 3.6)\si{\ohm}
\end{align}

\begin{table}
  \centering
  \caption{Daten zur Wheatstone-Brücke}
  \label{tab:WS}
  \sisetup{table-format=2.1}
  \begin{tabular}{c c c c}
  \toprule
  $R_2 \, [\si{\ohm}]$ &$R_3 \, [\si{\ohm}]$ &$R_4 \, [\si{\ohm}]$ & $R_x \, [\si{\ohm}]$\\
  \midrule
   664 & 425.25 & 574.75& 491.285\\
   1000 & 330 & 670 & 492.537 \\
   332 & 595.5 & 404.5 & 488.766 \\
  \bottomrule
  \end{tabular}
\end{table}
%\newpage
\subsection{Kapazitätsmessbrücke}
Für die Werte des verlustbehafteten Kondensators (Wert 9) ergibt sich also:
\begin{align}
  C_9=(193.19 \pm 58.88)\si{\nano} \si{\farad}\\
  R_9=(14.95 \pm 7.50)\si{\kilo} \si{\ohm}\\
\end{align}
Für den Wert des Kondensators (Wert 3) ergibt sich:
\begin{align}
  C_3=(434.94 \pm 8.32)\si{\nano} \si{\farad}
\end{align}
\subsection{Induktivitätsmessbrücke}
\subsection{Maxwell-Brücke}
\subsection{Wien-Robinson-Brücke}
\begin{figure}
  \centering
  \includegraphics{WRplot.pdf}
  \caption{Plot.}
  \label{fig:plot}
\end{figure}


Siehe \autoref{fig:plot}!
