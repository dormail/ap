\section{Diskussion}
\label{sec:Diskussion}
Im ersten Teil wurde ein Dämpfungswiderstand von $R = (82,8 \pm 0,9)\si{\ohm}$ gemessen.
Es gab keinen gegebenen Wert, mit dem man diesen vergleichen kann, aufgrund der geringen
Unsicherheit ist aber davon auszugehen, dass dieser akurat ist. Eine Fehlerquelle ist
allerdings der Widerstand der Schaltung selbst (der Drähte), welche den Widerstand leicht
nach oben verschiebt.
\\
Der Widerstand für den aperiodischen Grenzfalls wurde im zweiten Teil mit $R_\text{ap} =
\SI{1265}{\ohm}$ bestimmt. Er wies zum Theoriewert $R_\text{ap}^\text{theo} =
\SI{1673}{\ohm}$ eine relative Abweichung von $24,4\%$ auf. Der Fehler ist ziemlich niedrig,
dafür dass nur ein Durchlauf gemacht wurde. Die Präzision könnte aber weiter erhöht
werden, indem mehrere Messungen betrieben werden. Eine bessere Strategie wär es, sich dem
aperiodischen Grenzfall sowohl von oben als auch von unten zu nähern, um so eine
Verschiebung in eine Richtung zu verhindern.
\\
Im letzten Teil wurde die Breite der Filterkurve mit $b = \SI{1.5}{\kilo\hertz}$ bestimmt.
Der theoretisch bestimmt Wert liegt mit $b_\text{theo} = \SI{12.35}{\kilo\hertz}$ um
$723\%$ dadrüber, der relative Fehler liegt bei $87,8\%$. Der Fehler ist sehr groß, wobei
die Ursachen dafür verschieden sein können. Ein Problem kann sein, dass das
tatsächliche Maximum vom Amplitudenverhältnis nicht gemessen wurde, wodurch der
$\frac{1}{\sqrt{2}}$ viel zu niedrig seini kann. Das würde die enorme Verschiebung nach
unten erklären.

