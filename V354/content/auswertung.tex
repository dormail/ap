\section{Auswertung}
\label{sec:Auswertung}

\subsection{Zeitabhängigkeit der Amplitude beim Schwingfall}
\label{sec:Zeitabhängigkeit der Amplitude beim Schwingfall}
Im ersten Teil wurde im beim Schwingfall das exponentielle Abklingen der Amplitude beim
Schwingfall untersucht. Dabei wurden im festen Intervall von $\SI{13.6}{\micro\second}$
die Maxima bzw. Minima der Schwingung betrachtet, um so die momentane Schwingungsamplitude
zu erhalten. Dabei wurden folgende Messwerte gemessen:
\begin{table}
	\centering
	\caption{Messdaten zur Zeitabhängigkeit der Amplitude. Das Zeitintervall zwischen
	zwei Messpunkten folgt aus der Speisefrequenz mit $\Delta t =
\SI{13.6}{\micro\second}$.}
	\label{tab:Messwerte-5a}
	\sisetup{table-format=2.1}
	\begin{tabular}{c}
		\toprule
		$U / \si{\milli\volt}$\\
		\midrule
		372	\\
		308	\\
		276	\\
		232	\\
		196	\\
		168	\\
		144	\\
		116	\\
		100	\\
		84 	\\
		76 	\\
		64 	\\
		52 	\\
		48 	\\
		40 	\\
		36 	\\
		24 	\\
		20 	\\
		24 	\\
		16 	\\
		\bottomrule
	\end{tabular}
\end{table}
Da das Abklingen exponentiell stattfindet, liegt eine Funktion der Form
\begin{equation}
	U(t) = U_0 e^{-kt}
\end{equation}
für die Ausgleichsrechnung nahe. Mit der \texttt{curve\_fit}-Funktion aus scipy konnten
die Parameter
\begin{equation}
	U_0 = (436,2 \pm 3,5) \, \si{\milli\volt}
	\qquad
	k = (1,183 \pm 0,013) \cdot 10^4 \, \si{\second^{-1}}
\end{equation}
bestimmt werden. Die Messdaten sowie die Ausgleichskurve sind in \autoref{fig:5a-plot}
grafisch dargestellt.
\begin{figure}[H]
	\centering
	\includegraphics{build/5a.pdf}
	\caption{Messdaten und Curve Fit für die Zeitabhängigkeit der Amplitude. Die
		Messwerte stammen alternierend aus Wellenberg und -Tal, es wurden jedoch
	nur die Beträge betrachtet.}
	\label{fig:5a-plot}
\end{figure}
Damit lässt sich die Zerfallszeit $T_\text{ex}$ bestimmen, welche definiert ist durch
\begin{equation}
	U\left( T_\text{ex} \right) = \frac{U(0)}{e}.
\end{equation}
Diese lässt sich aus den Fitparametern bestimmen:
\begin{equation}
	T_\text{ex} = \frac{1}{k} = (8,45 \pm 0,09)10^{-5} \, \si{\second}
\end{equation}
Daraus lässt sich bei bekannter Induktivität der Dämpfungswiderstand $R$ bestimmen.
Erstere wurde durch $L = (3,5 \pm 0,01) \, \si{\milli\henry}$ gegeben, damit folgt der
Widerstand
\begin{equation}
	R = \frac{2L}{T_\text{ex}} = (82.8 \pm 0.9) \, \si{\ohm}.
\end{equation}
Für die Ungenauigkeiten wurde mit linearer Fehlerfortpflanzung gearbeitet.

\subsection{Bestimmung des Widerstandes für den aperiodischen Grenzfall}
\label{sec:Bestimmung des Widerstandes für den Aperiodischen Grenzfall}
Für den aperiodischen Grenzfall folgt mit den gegebenen Werten
\begin{equation}
	L = (3,5 \pm 0,01) \, \si{\milli\henry}
	\qquad
	C = (5 \pm 0,02) \, \si{\nano\farad}
\end{equation}
ein Theoriewert von
\begin{equation}
	R_\text{ap} = 2 \sqrt{\frac{L}{C}} = (1673 \pm 4) \, \si{\ohm}.
\end{equation}
Experimentell wurde ein Wert von
\begin{equation}
	R_\text{ap} = 1265 \, \si{\ohm}
\end{equation}
bestimmt. Dadraus folgt eine relative Abweichung von $24,4 \%$.

