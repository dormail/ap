\section{Diskussion}
\label{sec:Diskussion}

Im ersten Teil der Auswertung konnten realistische Werte für die Gegenspannungen im Bereich von knapp über $0V$ bis circa $1.5V$ ermittelt werden. Die geringen Abweichungen der Kurven von der erwarteten Form können neben obligatorischen Messunsicherheiten mit widrigen Umständen bei der Messung begründet werden. So konnte mit dem genutzten Versuchsaufbau keine optimale Abschirmung der Photozelle erreicht werden. Das Licht der Tichschlampe kann beispielsweise bereits einen Einfluss auf die Messwerte haben. 
\newline
Im Weiteren wurde über den Zusammenhang der ermittelten Gegenspannungen zu den jeweiligen Lichtfrequenzen ein Wert für $h/e_0$ berechnet. Der Theoriewert für den Quotienten $h/e_0$ beträgt gerundet $4.14*10^{-15} Vs$. In \autoref{sec:he} wurde der Quotient zu 
\begin{equation*}
    \frac{h}{e_0}=(3.6\pm0.4)*10^{-15} Vs
\end{equation*}
berechnet. Somit weicht der berechnete Wert um gerundet 13 \% von dem Theoriewert ab. Diese Abweichung ist relativ gering und das berechnete Konfidenzintervall von $[3.2 , 4.0]Vs$ liegt noch näher am Theoriewert. Die geringe verbleibende Abweichung kann mit den bereits beschriebenen Einflüssen auf die Messgenauigkeit erklärt werden. 