\section{Auswertung}
\label{sec:Auswertung}
\subsection{Bestimmung der Gegenspannung $U_g$}
Im ersten Teil der Auswertung des Versuchs wird die Gegenspannung mit einer linearen Ausgleichrechnung zu den gemessenen Strömen und Brems- bzw. Beschleunigungsspannungen bestimmt. In den einzelnen Unterabschnitten sind zu den 4 vermessenen Spektrallinien die Messdaten und die zugehörige Ausgleichsgerade in einer Tabelle sowie einem Diagramm dargestellt. Aus den Parametern der Ausgleichsgeraden werden dann nach $U_g=-b/a$ die Gegenspannungen als die Nullstellen der Ausgleichsgeraden (Form $y=ax+b$) berechnet.
\subsubsection{Gelbe Spektrallinie}
\begin{table}[H]
  \centering
  \caption{Messdaten zur gelben Spektrallinie, $\lambda=578nm$}
  \label{tab:gelb}
  \begin{tabular}{ll}
  \toprule
  U {[}V{]} & I {[}nA{]} \\
  \midrule
  20 & 0 \\
  15 & 0 \\
  10 & 0 \\
  5 & 0 \\
  2 & 0 \\
  1 & 0 \\
  0 & 0 \\
  -0.4 & 0.1 \\
  -0.5 & 0.2 \\
  -1 & 0.3 \\
  -1.5 & 0.4 \\
  -2 & 0.5 \\
  -2.5 & 0.6 \\
  -3 & 0.8 \\
  -3.5 & 0.9 \\
  -4 & 0.9 \\
  -4.5 & 1.3 \\
  -5.5 & 1.3 \\
  -7 & 1.2 \\
  -10 & 1.8 \\
  -14 & 1.8 \\
  -18.7 & 1.9 \\
  -19.14 & 2\\\bottomrule
  \end{tabular}
  \end{table}

  \begin{figure}[H]
    \centering
    \includegraphics{gelb.pdf}
    \caption{Messdaten und Ausgleichsgerade zur gelben Spektrallinie, Wurzel des Photostroms in Abhängigkeit von der Bremsspannung}
    \label{fig:gelb}
  \end{figure}

  \begin{table}[H]
    \centering
    \caption{Parameter der Ausgleichsgeraden und Gegenspannung zur gelben Spektrallinie}
    \label{tab:ugl}
    \begin{tabular}{lll}
      \toprule
      $a [\si{\sqrt{\nano\ampere}\per\volt}]$ &
      $b [\sqrt{\si{\nano\ampere}}]$ &
      $U_g [\si{\volt}]$ \\ \midrule
      -0.31 \pm 0.07     & 0.16 \pm 0.08  & 0.51 \pm 0.28   \\ \bottomrule
    \end{tabular}
    \end{table}

\subsubsection{Grüne Spektrallinie}
\begin{table}[H]
  \centering
  \caption{Messdaten zur grünen Spektrallinie, $\lambda=546nm$}
  \label{tab:gruen}
  \begin{tabular}{ll}
    \toprule
    U [V] & I [nA] \\ \midrule
    0     & 0      \\
    -0.22  & 0.2     \\
    -0.5   & 0.4     \\
    -1     & 0.6     \\
    -1.52  & 1      \\
    -2     & 1.2    \\
    -3     & 1.5    \\
    -4     & 1.8    \\
    -5     & 2      \\
    -6     & 2.2    \\
    -7     & 2.4    \\
    -10.2  & 2.7    \\
    -13    & 2.9    \\
    -16    & 3      \\
    -19.14 & 3.2    \\
  \bottomrule
  \end{tabular}
  \end{table}

  \begin{figure}[H]
    \centering
    \includegraphics{gruen.pdf}
    \caption{Messdaten und Ausgleichsgerade zur grünen Spektrallinie, Wurzel des Photostroms in Abhängigkeit von der Bremsspannung}
    \label{fig:gruen}
  \end{figure}

  \begin{table}[H]
    \centering
    \caption{Parameter der Ausgleichsgeraden und Gegenspannung zur grünen Spektrallinie}
    \label{tab:ugr}
    \begin{tabular}{lll}
      \toprule
      $a [\si{\sqrt{\nano\ampere}\per\volt}]$ &
      $b [\sqrt{\si{\nano\ampere}}]$ &
      $U_g [\si{\volt}]$ \\ \midrule
      -0.48 \pm 0.09     & 0.24 \pm 0.11  & 0.50 \pm 0.24   \\ \bottomrule
    \end{tabular}
    \end{table}

\subsubsection{Erste violette Spektrallinie}
\begin{table}[H]
  \centering
  \caption{Messdaten zur ersten violetten Spektrallinie, $\lambda=435nm$}
  \label{tab:violett1}
  \begin{tabular}{ll}
  \toprule
  U [V]& I [nA]\\
  \midrule
  0.7 & 0 \\
  0.54 & 0.05 \\
  0.38 & 0.1 \\
  0.1 & 0.2 \\
  0 & 0.35 \\
  -0.5 & 1 \\
  -1 & 1.4 \\
  -1.5 & 1.6 \\
  -2 & 2 \\
  -3 & 2.8 \\
  -4 & 3.2 \\
  -5 & 3.7 \\
  -6 & 4.1 \\
  -7.03 & 4.5 \\
  -9 & 4.9 \\
  -12 & 5.5 \\
  -15 & 5.6 \\
  -19.14 & 6.2\\\bottomrule
  \end{tabular}
  \end{table}

  \begin{figure}[H]
    \centering
    \includegraphics{v1.pdf}
    \caption{Messdaten und Ausgleichsgerade zur ersten violetten Spektrallinie, Wurzel des Photostroms in Abhängigkeit von der Bremsspannung}
    \label{fig:v1}
  \end{figure}

  \begin{table}[H]
    \centering
    \caption{Parameter der Ausgleichsgeraden und Gegenspannung zur ersten violetten Spektrallinie}
    \label{tab:uv1}
    \begin{tabular}{lll}
      \toprule
      $a [\si{\sqrt{\nano\ampere}\per\volt}]$ &
      $b [\sqrt{\si{\nano\ampere}}]$ &
      $U_g [\si{\volt}]$ \\ \midrule
      -0.55 \pm 0.06    & 0.56 \pm 0.04  &1.02 \pm 0.13   \\ \bottomrule
    \end{tabular}
    \end{table}

\subsubsection{Zweite violette Spektrallinie}
\begin{table}[H]
  \centering
  \caption{Messdaten zur zweiten violetten Spektrallinie, $\lambda=405nm$}
  \label{tab:violett2}
  \begin{tabular}{ll}
    \toprule
    U [V] & I [nA] \\ \midrule
    1.2  & 0      \\
    0.7  & 0.02    \\
    0.5  & 0.06    \\
    0.2  & 0.12    \\
    0     & 0.2     \\
    -0.5    & 0.2     \\
    -1     & 0.5     \\
    -2     & 1      \\
    -3     & 1      \\
    -4     & 1.4    \\
    -5     & 1.6    \\
    -6     & 1.8    \\
    -7     & 1.9    \\
    -8     & 2.2    \\
    -9     & 2.5    \\
    -10    & 2.6    \\
    -11    & 2.7    \\
    -12    & 2.7    \\
    -15    & 2.8    \\ 
    -19.11 & 3      \\ \bottomrule
  \end{tabular}
  \end{table}
\begin{figure}[H]
  \centering
  \includegraphics{v2.pdf}
  \caption{Messdaten und Ausgleichsgerade zur zweiten violetten Spektrallinie, Wurzel des Photostroms in Abhängigkeit von der Bremsspannung}
  \label{fig:v2}
\end{figure}

\begin{table}[H]
  \centering
  \caption{Parameter der Ausgleichsgeraden und Gegenspannung zur zweiten violetten Spektrallinie}
  \label{tab:uv2}
  \begin{tabular}{lll}
    \toprule
    $a [\si{\sqrt{\nano\ampere}\per\volt}]$ &
    $b [\sqrt{\si{\nano\ampere}}]$ &
    $U_g [\si{\volt}]$ \\ \midrule
    -0.29 \pm 0.04    & 0.373 \pm 0.028  & 1.29 \pm 0.22   \\ \bottomrule
  \end{tabular}
  \end{table}

\subsection{Bestimmung von $\frac{h}{e}$ und der Austrittsarbeit $A_k$}
\label{sec:he}
Das Verhältnis $\frac{h}{e}$ und die Austrittsarbeit $A_k$ lassen sich mit einer  linearen Ausgleichsrechnung zu den bereits ermittelten Gegenspannungen und den entsprechenden Lichtfrequenzen bestimmen. 
\begin{figure}[H]
  \centering
  \includegraphics{ug.pdf}
  \caption{Daten und Ausgleichsgerade zu den Gegenspannungen und den entsprechenden Frequenzen des Lichts}
  \label{fig:ug}
\end{figure}
\noindent
Die Steigung der Ausgleichsgeraden entspricht dabei dem Koeffizienten $h/e$ und der y-Achsenabschnitt entspricht der negativen Austrittsarbeit in $eV$. Auf diese Weise erhält man:
\begin{align}
  \frac{h}{e_0} &= (3.6\pm0.4)*10^{-15} Vs\\
  A_k &= (1.40\pm0.25) eV\\
\end{align}
\subsection{Kurvenverlauf}
In \autoref{fig:ui} wurde für die gelbe Spektrallinie ($\lambda=578nm$)der Photostrom in Abhängigkeit der anliegenden Spannung  für den gesamten vermessenen Spannungsbereich ($-20 $bis $+20 V$) dargestellt.  
\begin{figure}[H]
  \centering
  \includegraphics{UI.pdf}
  \caption{Photostrom in Abhängigkeit der anliegenden Spannung für die gelbe Spektrallinie ($\lambda=578nm$)}
  \label{fig:ui}
\end{figure}
\noindent
Die zu sehende Kurve entspricht dem erwarteten Kurvenverlauf. So nähert sich der Photostrom für hohe Beschleunigungsspannungen asymptotisch einem Sättigungswert an, geht dann in einen linear fallenden Bereich über und fällt gegen 0 für Spannungen, die sich der Gegenspannung $U_g$ annähern. Das Phänomen, dass sich der Photostrom einem Sättigungswert annähert ist damit zu erklären, dass die Lichtintensität konstant gehalten wurde. Aufgrund der konstanten Lichtintensität ist die Menge der durch den Photoeffekt ausgelösten Elektronen begrenzt. Die Beobachtung, dass sich der Photostrom bereits für $U<U_g$ dem Wert 0 annähert hängt damit zusammen, dass die Photoelektronen nicht monoenergetisch verteilt sind, sondern eine Energieverteilung aufweisen.