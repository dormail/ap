\section{Auswertung}
\label{sec:Auswertung}

\subsection{Zählrohr-Charakteristik}
\begin{figure}
  \centering
  \includegraphics{build/plot1.pdf}
  \caption{Plot.}
  \label{fig:plot}
\end{figure}


Siehe \autoref{fig:plot}!
\subsection{Abstand zwischen Primär- und Nachentladungsimpulsen}
Im versuch wurde das in zu sehende Oszilloskopbild aufgenommen. Als Abstand zwischen Primär- und Nachentladungsimpulsen wurde hier $450 \mu s$ abgelesen.


\subsection{Bestimmung der Totzeit}
Oszilloskop-Methode
2-Quellen Methode
\begin{table}
  \centering
  \caption{Zählraten zur 2-Quellen-Methode}
\label{tab:mess2}
  \sisetup{table-format=2.1}
  \begin{tabular}{c c c c}
  \toprule
  Quelle & $N [\frac{Imp}{120s}]$ \\
  \midrule
  1     & 96041  \\
  2     & 76518 \\
  1 + 2 & 158479 \\
  \bottomrule
  \end{tabular}
  \end{table}
  \subsection{Pro Teilchen vom Zählrohr freigesetzte Ladungsmenge}
  