\section{Diskussion}
\label{sec:Diskussion}
Anfangs wurde das Kontaktpotential
\[
	K = \SI{2.261}{\volt}
\]
bestimmt. Dafür gibt es keinen Literaturwert, es sind jedoch mehrere Fehlerquellen
ersichtlich. Hauptsächlich ist dabei der XY-Schreiber, auf den später noch eingegangen
wird.
\\
Im letzten Teil wurde eine Anregungsenergie für Quecksilber von 
\[
	E = (5,51 \pm 0,495) \, \si{eV}
\]
berechnet. Der Literaturwert dazu lautet \cite{leifi}
\[
	E^\text{lit} = 4,9 \, \si{eV}.
\]
Die Fehlerquellen können dabei von verschiedenen Stellen ausgehen: So können mechanische
Störungen den XY-Schreiber beeinflusst haben. Auch ist die Filzschreiberlinie relativ
dick, wodurch kein exaktes Ablesen möglich ist. Für eine exaktere Messung wäre eine digitale
Messapperatur von Vorteil.
\\
Die elastischen Stöße können hierbei vernachlässigt werden, da bei dem großen
Massenunterschied
\[
	m(e^-) \ll m(Ag)
\]
praktisch kein signifikanter Energieübertrag stattfindet.

