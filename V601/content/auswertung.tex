\section{Auswertung}
\label{sec:Auswertung}

\subsection{Mittlere freie Weglänge}
\label{sec:Mittlere freie Weglänge}
Als allererstes soll die mittlere freie Weglänge der Elektronen in der Apperatur berechnet
werden. Mit der Formel
\[
	w / \si{\centi\meter} = \frac{0,0029}{p_\text{sät}} = \frac{0,0029}{5,5 \cdot 10^7
	\cdot \exp(-6876 / T)}
\]
folgen die Werte in \autoref{tab:mfp}. Zum Vergleich wird die Anzahl an Stößen auf einer
Distanz von $\SI{1}{\centi\meter}$ angegeben.
\begin{table}
	\centering
	\caption{Mittlere freie Weglänge der Elektronen bei den hier relevanten
	Temperaturen.}
	\label{tab:mfp}
	\sisetup{table-format=2.1}
	\begin{tabular}{c c c}
		\toprule
		$T / \si{^\circ C}$ & $\overline{w} / \si{\centi\meter}$ & $n \cdot
		\si{\centi\meter}$ \\
		\midrule
		 25    & 0,55 & 1,81 \\
		 148   & $6,53 \cdot 10^{-4}$ & $1,53 \cdot 10^{3}$ \\
		 168   & $3,12 \cdot 10^{-4}$ & $3,21 \cdot 10^{3}$ \\
		 197,5 & $1,17 \cdot 10^{-4}$ & $8,53 \cdot 10^{3}$ \\
		\bottomrule
	\end{tabular}
\end{table}

