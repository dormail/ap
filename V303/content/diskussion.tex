\section{Diskussion}
\label{sec:Diskussion}

\subsection{Diskussion der Phasenahängigkeit}
\label{sec:Diskussion der Phasenahängigkeit}
Im ersten Teil wurde der Zusammenhang zwischen Referenzphase und Amplitude untersucht. Die
Messdaten (\autoref{tab:messdaten-phase}) sind konsistent mit der Periodizitätsbedingung
\begin{equation}
	U(\phi) = U(\phi + 2\pi) \approx -U(\phi + \pi).
\end{equation}
Dies spiegelt sich auch im Fitparameter $b = 0,96\pm0,12 \approx 1$ wieder. Problematisch
stellte sich die Identifikation eines Cosinus dar. Fit- und Theoriekurve sind um einen
Phasenfaktor von $\Delta \phi \approx \pi/4$ verschoben. Dies lässt sich auf den
Koeffizient $c = 0,86 \pm 0,49 \approx \pi / 4$ schieben. Um aus dem Sinus der
Fitfunktion einen Cosinus zu erhalten wäre fast das doppelte $c_\text{Theo} = \pi/2 \approx
1,57$ nötig gewesen.
\\
Die Skalenfaktoren in y-Richtung $a$ und $d$ stimmen mit der optischen Beobachtung einer
geringen y-Verschiebung überrein. In den Fitparametern wird das durch das
Verhältniss der Parameter mit $d \ll b$ deutlich.
\\
Für eine exaktere Bestimmung wären mehr Messwerte nötig gewesen, wobei dann jedoch das
Aliasing problematisch werden würde.

\subsection{Diskussion zur Diodenintensität}
\label{sec:Diskussion zur Diodenintensität}
Im dritten Teil wurde der Abfall der Diodenintensität in Abhängigkeit von der
Beobachterdistanz $d$ untersucht. Dabei wurden die zwei Funktionen
\begin{equation}
	U(d) = = \frac{a}{d} + b,
	\qquad
	a = (637 \pm 51,2) \, \frac{\si{\volt}}{\si{\centi\meter}}
	\quad
	b = (-20,7 \pm 3,5) \, \si{\volt}
\end{equation}
und 
\begin{equation}
	g(d) = \frac{a^\prime}{d^2} + b^\prime
	\qquad
	a^\prime = (4988 \pm 147,4) \, \frac{\si{\volt}}{\si{\centi\meter}}
	\quad
	b^\prime = (-3 \pm 0,82) \, \si{\volt}
\end{equation}
berechnet. Optisch ist die zweite Kurve sehr viel näher an den Messwerten in
\autoref{fig:plot-diode} dran. Auch sind die relativen Unsicherheiten sehr viel kleiner.
Damit konnte die $1/x$-Gesetzmäßigkeit experimentell nicht bestätigt werden.
\\
Auffällig sind bei beiden Fits die Werte $b < b^\prime < 0$. Die entspräche einem Raum mit
negativem Untergrund, welcher offensichtlich nicht möglich ist. Die Messung sollte
wahrscheinlich unter besseren Bedingungen (Raum verdunkeln) durchgeführt werden, um
mögliche Fehlerquellen auszuschließen. 

