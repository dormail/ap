\section{Auswertung}
\label{sec:Auswertung}

\subsection{Untersuchung der Phasenabhängigkeit der Ausgangsspannung}
\label{sec:Untersuchung der Phasenabhängigkeit der Ausgangsspannung}
Im ersten Teil wurde der Zusammenhang zwischen Ausgangsspannung und der Phase der
Referenzspannung untersucht. Dabei wurden sieben Messwerte aufgenommen, welche in
\autoref{fig:plot-phase} dargestellt sind.
\begin{figure}
	\centering
	\includegraphics{phase.pdf}
	\caption{Messdaten und Curve Fit für den Zusammenhang zwischen Ausgangsspannung und
	Phase der Referenzspannung.}
	\label{fig:plot-phase}
\end{figure}

\subsection{Messung der Lichtintensität einer LED}
\label{sec:Messung der Lichtintensität einer LED}
Mit der in \autoref{sec:Durchführung} beschriebenen Apparatur wurden die in 
\autoref{tab:messdaten-led} aufgenommen. Aufgrund der Ausbreitung der Lichtwellen wird
hier eine Fitfunktion der Form
\begin{equation}
	f(d) = \frac{a}{d} + b
\end{equation}
verwendet.\\
Der Curve Fit aus scipy hat dabei die Parameter
\begin{equation}
	a = (637 \pm 51.2) \frac{\si{\volt}}{\si{\centi\meter}}
	\qquad
	b = (-20.7 \pm 3.5) \si{\volt}
\end{equation}
gefunden.
\begin{table}
	\centering
	\caption{Messergebnisse zur Lichtintensität der LED.}
	\label{tab:messdaten-led}
	\sisetup{table-format=2.1}
	\begin{tabular}{c c}
		\toprule
		$d / (\si{\centi\meter})$ & $U / \si{\volt}$ \\
		\midrule
		9,55         &           54 \\
		11           &           38 \\
		12           &           30 \\
		13           &           26 \\
		14           &           20 \\
		15           &           18 \\
		17           &           16 \\
		20           &           10 \\
		25           &           5  \\
		30           &           3  \\
		35           &           2  \\
		\bottomrule
	\end{tabular}
\end{table}
Die Messdaten sowie Ausgleichskurve sind in \autoref{fig:plot-diode} dargestellt.
\begin{figure}
	\centering
	\includegraphics{diode.pdf}
	\caption{Messdaten und Curve Fit der Lichtintensität in Abhängigkeit von der
	Distanz.}
	\label{fig:plot-diode}
\end{figure}
