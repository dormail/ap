\section{Auswertung}
\label{sec:Auswertung}
\subsection*{a) Darstellung der Temperaturverläufe}
Im ersten Schritt der Auswertung des Versuches wurden die gemessenen Temperaturverläufe graphisch dargestellt.
In \autoref{fig:plot} 
wurden die Daten in einem Zeit-Temperatur Diagramm geplottet. Die zur Verfügung gestellten Messdaten sind in einer Tabelle 
im Anhang zu finden.

\begin{figure}
  \centering
  \includegraphics{plot.pdf}
  \caption{Gemessene Temperaturverläufe}
  \label{fig:plot}
\end{figure}
\newpage
\subsection*{b) Nicht-lineare Ausgleichsrechnung}
Als nächtes wurden die Messdaten durch die Funktion $T(t)=A\cdot t^2+B\cdot t+C$ approximiert.
Mit der CurveFit Funktion und Fehlerrechnung in Python wurden dazu folgende Parameter ermittelt: 

\begin{align*}
  \intertext{Für T1:}
  A_1 &= \SI{-0.011610+-0.000151}{\celsius\per\minute\squared}\\
  B_1 &= \SI{1.216790 +- 0.005466}{\celsius\per\minute}\\
  C_1 &= \SI{21.820081 +- 0.041345}{\celsius}
\end{align*}
  
\begin{align*}  
  \intertext{Für T2:}
  A_2 &= \SI{0.003438 +- 0.000962}{\celsius\per\minute\squared}\\
  B_2 &= \SI{-0.672523 +- 0.034819}{\celsius\per\minute}\\
  C_2 &= \SI{22.720187 +- 0.263361}{\celsius}
\end{align*} 

In \autoref{fig:plotb} 
sind die gefitteten Funktionen zusammen mit den Messdaten zu sehen . 

\begin{figure}
  \centering
  \includegraphics{plot1.pdf}
  \caption{Nichtlineare Approximation der Temperaturverläufe}
  \label{fig:plotb}
\end{figure}
%\newpage
\subsection*{c) Bestimmung der Differentialquotienten}
Im nächsten Schritt sollten zu 4 verschiedenen Temperaturen die Differentialquotienten
$dT1/dt$ und $dT2/dt$ bestimmt werden. Dazu wurden die ermittelten Approximationsfunktionen differenziert. Im Allgemeinen
ergibt die Differenzierung die Funktion $T´(t)=2A\cdot t+B$. Im Speziellen ergeben sich also die Funktionen:
\begin{align*}  
  T1´ &= -0.024\cdot t+1.217\\
  T2´ &= 0.006\cdot t-0.673 \\
\end{align*} 

Die berechneten Werte für die Differentialquotienten für vier Temperaturen
sind in Tabelle \ref{tab:Diff} aufgelistet. Hier wurden die Fehlerwerte aus den Fehlern der einzelnen Parameter nach
\begin{equation*}
  \symup{\Delta} \frac{\symup{d}T}{\symup{d}t} = \sqrt{(2t\cdot\symup{\Delta} A)² + (\symup{\Delta} B)²}
  \end{equation*}
  berechnet. Gewählt wurden die Temperaturen bei 8, 16, 24 und 32 Minuten. 
\begin{table}
\centering
\caption{Differentialquotienten}
\label{tab:Diff}
\sisetup{table-format=2.1}
\begin{tabular}{c c c c c}
\toprule
$t \, [\si{\minute}]$ &$T1 \, [\si{\celsius}]$ &$T2 \, [\si{\celsius}]$ & $\frac{\symup{d}T_1}{\symup{d}t} \,[\si{\celsius\per\minute}]$ & 
$\frac{\symup{d}T_2}{\symup{d}t} \,[\si{\celsius\per\minute}]$\\
\midrule
 8 & 30.9&17.7&1.025 \pm \,0.006 & -0.625 \pm \,0.038\\
 16 & 38.4&12.4&0.833 \pm \,0.007 & -0.577 \pm \,0.046\\
 24 & 44.3&8.7&0.641 \pm \,0.009 & -0.529 \pm \,0.058\\
 32 & 48.9&4.3&0.449 \pm \,0.011 & -0.481 \pm \,0.071\\
\bottomrule
\end{tabular}
\end{table}
\newpage
\subsection*{d) Bestimmung der Güteziffern}
Aus den errechneten Werten für die Differentialquotienten kann nun die Güteziffer der benutzten Wärmepumpe 
ermittelt werden. Anschließend wird dieses Ergebnis mit der Güteziffer einer idealen Wärmepumpe verglichen.
Im Theorieteil wurde bereits beschrieben, dass die reale Güteziffer nach  den Gleichungen (\ref{eqn:bestgueteziffer1}) 
und (\refeq{eqn:messgueteziffer}) bestimmt werden kann.
 
 Dabei ist die spezifische Wärmekapazität des Wassers $c_w = \SI{4.187}{\joule\per\gram\per\kelvin}$, die Masse des Wassers 
 $m_1 = \SI{4}{\kilogram}$ und die Wärmekapazität 
 der Kupferschlangen ist gegeben als $c_\text{k} m_\text{k}$ = \SI{750}{\joule\per\kelvin}.
 Die ideale Güteziffer wurde hier nach Gleichung (\ref{eqn:gueteideal}) bestimmt. In Tabelle \ref{tab:Güte} wurden die so errechneten Werte 
 für die reale Güteziffer $\nu$ und die
 ideale Güteziffer $\nu_\text{id}$ angegeben. Zur besseren Vergleichbarkeit wurde außerdem das Verhältnis der beiden Werte aufgelistet. 
 
 \begin{table}
 \centering
 \caption{Güteziffern}
 \label{tab:Güte}
 \sisetup{table-format=2.1}
 \begin{tabular}{c c c c}
 \toprule
 $t \, [\si{\minute}]$ &$\nu \,$ &$\nu_\text{id} \,$ & $\frac{\nu_\text{id}}{\nu} \,$  \\
 \midrule
  8 & 2.49 \pm \,0.02 & 23.03 & 9.2 \\
  16 & 2.02 \pm \,0.02 & 11.98 &  5.9\\
  24 & 1.56 \pm \,0.03 & 8.92 & 5.7 \\
  32 & 1.09 \pm \,0.04 & 7.22 & 6.6 \\
 \bottomrule
 \end{tabular}
 \end{table}
 
An der Tabelle ist direkt zu erkennen, dass die ermittelten Werte für $\nu $ und $\nu_\text{id}$ 
stark voneinander abweichen. So ist $\nu_\text{id}$ für die betrachteten Werte zwischen 5.7 und 9.2 mal so hoch wie $\nu $.

 \subsection*{e) Bestimmung des Massendurchsatzes}
 Wie in 1.3.2. beschrieben kann der Massendurchsatz mit den Formeln (\ref{eqn:waermeleitung2}) 
 und (\refeq{eqn:massendurchsatz}) ermittelt werden. Die Masse des zweiten Reservoirs ist dabei gegeben als $m_2=m_1=4kg$.
 Zusätzlich wird für die Bestimmung des Massendurchsatzes die Verdampfungswärme L benötigt. Diese Verdampfungswärme L eines 
 Stoffes kann man aus seiner Dampfdruck-Kurve gewinnen. Dazu muss eine Ausgleichsrechnung zu der zugehörigen Dampfdruck-Kurve 
 durchgeführt werden. Hier wurde eine lineare Ausgleichsrechnung zu den Werten ln(p2) und 1/T2 durchgeführt. Die Messdaten und die 
 Ausgleichsgerade wurden in \autoref{fig:plote} dargestellt.
 \begin{figure}
  \centering
  \includegraphics{plot2.pdf}
  \caption{Ausgleichsrechnung zur Bestimmung der Verdampfungswärme}
  \label{fig:plote}
\end{figure}
 
 Als Parameter der Ausgleichsgeraden(ax+b) ergeben sich $a=-1719.830 \pm 88.369$ und $b=7.327 \pm 0.310$.
 Nach  $L_\text{reg} = -a \cdot R$, mit 
 $R= \SI{8.3144621}{\joule\per\mol\per\kelvin}$ als der universellen Gaskonstanten  
 lässt sich L also bestimmen:
 
 \begin{equation*}
 L= \SI{14300+-734}{\joule\per\mol}
 \end{equation*}
 
 Umgerechnet in Gramm mit der molaren Masse von Dichlordifluormethan $M=120,91 g/mol$ ergibt sich:
 
 \begin{equation*}
 L = \SI{118+-6}{\joule\per\gram}
 \end{equation*}
 
 Mit den bereits bestimmten Differentialquotienten lassen sich nun die Massendurchsätze nach den Formeln (\ref{eqn:waermeleitung2}) 
 und (\refeq{eqn:massendurchsatz}) berechnen. Im Folgenden wurden die Massendurchsätze zu den verschiedenen Zeiten aufgelistet:
 \begin{table}
  \centering
  \label{tab:massendurchsatz}
  \sisetup{table-format=2.1}
  \begin{tabular}{c c}
  \toprule
  $t \,[min] $ & $\frac{\Delta m}{\Delta t} [\si{\frac{g}{min}}]$  \\
  \midrule
  8  & -92.68 $\pm$ 4.8 \\
  16 & -85.56 $\pm$ 4.2 \\
  24 & -78.44 $\pm$ 3.6 \\
  32 & -71.33 $\pm$ 3.6 \\
  \bottomrule
  \end{tabular}
  \caption{Massendurchsätze}
  \end{table} 
  \newpage
  \subsection*{f) Bestimmung der mechanischen Kompressorleistung}
  Nun sollte die mechanische Leistung des Kompressors, die dieser abgibt, wenn
  er zwischen den Drücken p1 und p2 arbeitet, für die 4 bereits betrachteten Temperaturen berechnet werden. In Abschnitt 
  1.3.3 wurde beschrieben wie die mechanische Kompressorleistung $N_\text{mech}$ mit \autoref{eqn:kompressorleistung} 
  berechnet werden kann. Dazu muss $\rho$ aus der idealen Gasgleichung und den zur Verfügung gestellten Daten
  für Cl2F2C ($ρ_0= 5,51 g/l$ bei $T = 0°C$ und $p = 1 Bar, κ = 1,14$)bestimmt werden. 
  Zur Bestimmung von $\rho$ ergibt sich in unserem Versuch:
  
  \begin{equation*}
  \rho = \frac{p_\text{2} \cdot \rho_\text{o} \cdot T_\text{o}}{p_\text{o} \cdot T_2}.
  \end{equation*}
  Aus den beschriebenen Gleichungen und dem bereits berechneten Massendurchsatz lässt sich dann die mechanische Kompressorleistung 
  zu folgenden Werten bestimmen:

  \begin{table}
    \centering
    \caption{Mechanische Kompressorleistung}
    \label{tab:Kompressor}
    \sisetup{table-format=2.1}
    \begin{tabular}{c c c}
    \toprule
    $t \,[min]$ & $\rho \,[\si{\gram\per\liter}]$ & 
    $ N_\text{mech} \,[\SI{e-3}{\watt}]$\\
    \midrule
     8 & 21.73 & -5.01 \pm \,0.13\\
     16 & 18.97 & -10.02 \pm \,0.16\\
     24 & 18.16 & -14.09 \pm \,0.17\\
     32 & 17.36 & -17.15 \pm \,0.16\\
    \bottomrule
    \end{tabular}
    \end{table}