\section{Theorie}
\label{sec:Theorie}

In diesem Versuch untersuchen wir den Energietransport zwischen 
zwei Wärmereservoiren, mit dem Ziel die Kenngrößen der 
eingesetzten Wärmepumpe zu ermitteln.

\subsection{Das Prinzip der Wärmepumpe}
Aus Beobachtung von Prozessen in der Natur kann man sagen, dass Wärmeenergie immer vom heißeren zum kälteren
Körper fließt. Durch externe (mechanische) Arbeit kann man diesen Fluß jedoch umkehren. \\
Dabei nimmt nach 
dem ersten Hauptsatz der Wärmelehre das wärmere Reservoir die Wärmeenergie $Q_1$ auf, welche sich aus der
vom kälteren Reservoir entnommenen Wärmemenge $Q_2$, sowie der aufgewandten Arbeit $A$ zusammen:
\begin{equation}
	Q_1 = Q_2 + A
	\label{eqn:waermemenge}
\end{equation}

Davon ausgehend lässt sich die Güteziffer der Wärmepumpe als der Quotient von abgegebener Wärme
und aufgewandter Arbeit definieren:
\begin{equation}
	\nu = \frac{Q_1}{A}
	\label{eqn:gueteziffer}
\end{equation}

Aus dem zweiten Hauptsatz der Thermodynamik lässt sich dabei eine weitere Beziehung herleiten:
Ändert sich die Temperatur zwischen den Reservoiren nicht, so besagt dieser, dass
die Summe der sogenannten 
reduzierten Wärmemengen $\int \frac{\symup{d}Q}{T}$ verschwindet. Dies bedeutet für uns
\begin{equation}
	\frac{Q_1}{T_1} - \frac{Q_2}{T_2} = 0.
	\label{eqn:reduzierte}
\end{equation}

Dabei muss jedoch beachtet werden, dass \eqref{eqn:reduzierte} nur für ideale reversible 
Prozesse gilt. Dies kann von der technischen Realisierung der Wärmepumpe natürlich nicht erreicht werden.
Für den realistischen, irreversiblen Fall gilt die Beziehung
\begin{equation}
	\frac{Q_1}{T_1} - \frac{Q_2}{T_2} > 0.
	\label{eqn:reduzreal}
\end{equation}

\subsection{Die Arbeitsweise der Wärmepumpe}

\subsection{Die Bestimmung der Kenngrößen einer realen
Wärmepumpe}

\subsubsection{Bestimmung der realen Güteziffer}
\subsubsection{Bestimmung des Massendurchsatzes}
\subsubsection{Bestimmung der mechanischen Kompressorleistung 
$N_\text{mech}$}

% \cite{sample}
