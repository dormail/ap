\section{Diskussion}
\label{sec:Diskussion}

\subsection{Filterkurve und Güteziffer des Frequenzfilters}
\label{sec:Filterkurve und Güteziffer des Frequenzfilters}

Die gemessene Güte von $Q = 3,48 \pm 0,5$ weicht sehr stark von den eingestellten 
$Q_\text{Real} = 50$ ab. Das kann unteranderem damit erklärt werden, dass beim Maximum ($f
\approx \SI{35}{\kilo\hertz}$) nicht genug Messdaten aufgenommen wurden, wodurch die
gesamte Höhe der Kurve nicht deutlich wurde. Ein Problem bei der Messung war der
Funktionengenerator, bei dem das Signal nicht wirklich in der Frequenz regulierbar war,
weil der Regler zu grob und sehr fehleranfällig war. Eine Fehlbedienung des Filters durch die
Experimentierenden ist auch nicht auszuschließen. Zuletzt könnte auch der Filterverstärker
kaputt sein, wie sich bei der Messung herausstellte ist beim verwendeten Gerät die Einstellung 
$Q = 100$ schon nicht wer intakt gewesen.

\subsection{Magnetische Suszeptibilitäten}
\label{sec:Magnetische Suszeptibilitäten}

Die experimentellen Ergebnisses zu den magnetischen Suszeptibilitäten waren ähnlich
schlecht. Für $\symup{Dy_2O_3}$ wurden in \autoref{sec:Auswertung} drei Werte berechnet:
\begin{equation}
	\label{eqn:ergebnisse-Dy2O3}
	\chi_\text{theo} = 0,014
	\quad
	\chi_\text{U} = 0,069 \pm 0,004
	\quad
	\chi_\text{R} = 0,00987 \pm 0,00011
	\quad
\end{equation}
Mit den gleichen Methoden wurden folgende Werte für $\symup{Gd_2O_3}$ bestimmt:
\begin{equation}
	\label{eqn:ergebnisse-Gd2O3}
	\chi_\text{theo} = 0,026
	\quad
	\chi_\text{U} = 0,0292 \pm 0,0017
	\quad
	\chi_\text{R} = 0,00508 \pm 0,0002
	\quad
\end{equation}

Dabei ist auffällig, dass der mit der Spannungsmethode errechnete Wert grundsätzlich 
über dem Theoriewert liegt, die Ergebnisse der Widerstandsmethode grundsätzlich niedriger
als der theoretische Wert waren. Beim $\symup{Gd_2O_3}$ konnte mit der Spannungsmethode
ein relativ genauer Wert bestimmt werden (relativer Fehler $\approx 11\%$),
was aber auch pures Glück sein kann, da die anderen Ergebnisse sehr starke
Abweichungen voneinander vorweisen. Für genauere Ergebnisse wären
wahrscheinlich mehr als drei Durchläufe pro Probe und eine weniger antiquitierte Messapparatur 
nötig gewesen.

