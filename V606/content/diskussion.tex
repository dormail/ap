\section{Diskussion}
\label{sec:Diskussion}

\subsection{Filterkurve und Güteziffer des Frequenzfilters}
\label{sec:Filterkurve und Güteziffer des Frequenzfilters}

Die gemessene Güte von $Q = 4,7 \pm 0,4$ weicht sehr stark von den eingestellten 
$Q_\text{Real} = 50$ ab. Auch sind die beiden Kurven in \autoref{fig:filterkurve} optisch
sehr verschieden. Das kann unteranderem damit erklärt werden, dass beim Maximum ($f
\approx \SI{35}{\kilo\hertz}$) nicht genug Messdaten aufgenommen wurden, wodurch die
gesamte Höhe der Kurve nicht deutlich wurde. Ein Problem bei der Messung war der
Funktionengenerator, bei dem das Signal nicht wirklich in der Frequenz regulierbar war,
weil der Regler zu grob und sehr fehleranfällig war. Zusätzlich ist die Amplitude der
Eingangsspannung nicht bekannt gewesen, diese wurde anhand der Messwerte geschätzt.
Eine Fehlbedienung des Filters durch die
Experimentierenden ist auch nicht auszuschließen. Zuletzt könnte auch der Filterverstärker
kaputt sein, wie sich bei der Messung herausstellte ist beim verwendeten Gerät die Einstellung 
$Q = 100$ schon nicht mehr intakt gewesen. 

\subsection{Magnetische Suszeptibilitäten}
\label{sec:Magnetische Suszeptibilitäten}

Für $\symup{Dy_2O_3}$ wurden in \autoref{sec:Auswertung} drei Werte berechnet:
\begin{equation}
	\label{eqn:ergebnisse-Dy2O3}
	\chi_\text{theo} = 0,026
	\quad
	\chi_\text{U} = 0,164 \pm 0,009
	\quad
	\chi_\text{R} = 0,0234 \pm 0,00026
	\quad
\end{equation}
Mit den gleichen Methoden wurden folgende Werte für $\symup{Gd_2O_3}$ bestimmt:
\begin{equation}
	\label{eqn:ergebnisse-Gd2O3}
	\chi_\text{theo} = 0,014
	\quad
	\chi_\text{U} = 0,068 \pm 0,004
	\quad
	\chi_\text{R} = 0,0118 \pm 0,0005
	\quad
\end{equation}
Dabei konnte mit der Widerstandsmethode eine Abweichung von $10\%$ ($\symup{Dy_2O_3}$) bzw.
$15\%$ ($\symup{Gd_2O_3}$) erreicht werden. Die Ergebnisse der Spannungsmethode sind
schlechter; die Abweichung liegt bei beiden Messreihen über $300\%$. \\
Für genauere Ergebnisse wären wahrscheinlich mehr als drei Durchläufe pro Probe und eine 
weniger antiquierte Messapparatur nötig gewesen.

