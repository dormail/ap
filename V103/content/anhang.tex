\section{Anhang}
\label{sec:anhang}

\subsection{Ergebnisse der Versuchsvorbereitung}
Kennzahlen zu Stab 1
\begin{equation}
	m_1 = (394 \pm 0.1) \si{\gram}
	\qquad
	l_1 = 60.1 \si{\centi\meter}
	\qquad
	d_2 = 1 \si{\centi\meter}
\end{equation}
Kennzahlen zu Stab 2
\begin{equation}
	m_2 = (381.1 \pm 0.1) \si{\gram}
	\qquad
	l_1 = 55.15 \si{\centi\meter}
	\qquad
	d_2 = 1 \si{\centi\meter}
\end{equation}
Gewicht zur Belastung der Stäbe
\begin{equation}
	m = (949.9 \pm 0.1) \si{\gram}
\end{equation}

\newpage
\subsection{Ergebnisse der drei Messreihen}
\begin{table}
	\centering
	\caption{Messdaten für einen beidseitig aufliegenden Stab, $x$ beschreibt
	die Position der Messuhr, wobei $0\si{\centi\meter}$
	dem rechten Auflagepunkt, $28.5\si{\centi\meter}$ der Mitte bzw. dem Auflagepunkt 
	des Gewichts entspricht und $57\si{\centi\meter}$ linken Auflagepunkt entspricht.
	$D$ beschreibt die Auslenkung des Stabes an der Stelle $x$ bei Belastung.}
	\sisetup{table-format=2.1}
	\begin{tabular}{c c c}
	\toprule
	% ueberschriften der einzelnen spalten}
	Messuhr &
	$x / \si{\centi\meter}$ &
	$D(x) / \si{\micro\meter}$
	\\
	\midrule
	1 & 47 & 200 \\
	1 & 20 & 200 \\
	1 & 23 & 310 \\
	1 & 9 & 95 \\
	1 & 13 & 160 \\
	1 & 15 & 185 \\
	2 & 40 & 285.5 \\
	2 & 35 & 360 \\
	2 & 31 & 380 \\
	2 & 42 & 275 \\
	2 & 38 & 210 \\
	2 & 64 & 130 \\
	\bottomrule
\end{tabular}
\end{table}

\begin{table}
	\centering
	\caption{Messdaten für Stab 1, einseitig eingespannt. $x$ beschreibt
	die Position der Messuhr, wobei $0\si{\centi\meter}$
	dem Ort bei der Einspannung entspricht.
	$D$ beschreibt die Auslenkung des Stabes an der Stelle $x$ unter Last.}
	\sisetup{table-format=2.1}
	\begin{tabular}{c c c}
	\toprule
	% ueberschriften der einzelnen spalten}
	Messuhr &
	$x / \si{\centi\meter}$ &
	$D(x) / \si{\micro\meter}$
	\\
	\midrule
	% messuhr 1
	1 & 5 & 180 \\
	1 & 14 & 1120 \\
	1 & 18 & 2750 \\
	1 & 22 & 2500 \\
	1 & 26 & 3310 \\
	1 & 30 & 4200 \\
	1 & 34 & 5100 \\
	1 & 38 & 6010 \\
	1 & 16 & 1420 \\
	1 & 39 & 6010 \\
	% messuhr 2
	2 & 12 & 850 \\
	2 & 20 & 2100 \\
	2 & 24 & 2880 \\
	2 & 28 & 3800 \\
	2 & 32 & 4680 \\
	2 & 36 & 5600 \\
	2 & 40 & 6600 \\
	2 & 44 & 7600 \\
	2 & 42 & 7360 \\
	2 & 48 & 7500 \\
	\bottomrule
\end{tabular}
\end{table}

\begin{table}
	\centering
	\caption{Messdaten für Stab 2, einseitig eingespannt. $x$ beschreibt
	die Position der Messuhr, wobei $0\si{\centi\meter}$
	dem Ort bei der Einspannung entspricht.
	$D$ beschreibt die Auslenkung des Stabes an der Stelle $x$ unter Last.}
	\sisetup{table-format=2.1}
	\begin{tabular}{c c c}
	\toprule
	% ueberschriften der einzelnen spalten}
	Messuhr &
	$x / \si{\centi\meter}$ &
	$D(x) / \si{\micro\meter}$
	\\
	\midrule
	% messuhr 1 \\
	1 & 4 & 90 \\
	1 & 6 & 175 \\
	1 & 8 & 280 \\
	1 & 16 & 965 \\
	1 & 18 & 1190 \\
	1 & 20 & 1420 \\
	1 & 28 & 2600 \\
	1 & 30 & 2870 \\
	1 & 32 & 3190 \\
	1 & 40 & 4480 \\
	% messuhr 2 \\
	2 & 10 & 410 \\
	2 & 12 & 570 \\
	2 & 14 & 760 \\
	2 & 22 & 1700 \\
	2 & 24 & 1990 \\
	2 & 26 & 2280 \\
	2 & 34 & 3500 \\
	2 & 36 & 3880 \\
	2 & 38 & 4200 \\
	2 & 46 & 5580 \\
	\bottomrule
\end{tabular}
\end{table}
