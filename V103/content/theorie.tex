\section{Theorie}
\label{sec:Theorie}

\subsection{Berechnung des Flächentragheitmoments}
\label{sec:flaechentraegheitsmoment}
Das Flächenträgheitsmoment $\mathbf{I}$ wird, da in dem Versuch nur runde Stäbe betrachtet wurden,
in Polarkoordinaten berechnet:
\begin{align}
	\mathbf{I} 
	&= \int_Q y^2 dq(y) 
	\\
	&= \int_0^{2\pi} d\varphi \int_0^r dr^\prime r^\prime 
	\left( r^\prime\sin(\varphi) \right)^2
	\\
	&= \frac{\pi}{4} r^4
\end{align}

\cite{sample}
