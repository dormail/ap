\section{Auswertung}
\label{sec:Auswertung}
\subsection{Flächenträgheitsmoment der Stäbe}
Ausmessen mit der Schieblehre ergab für beide Stäbe einen Radius von 
$0.5\si{\centi\meter}$, woraus nach \autoref{sec:flaechentraegheitsmoment} ein
Flächenträgheitsmoment von

\begin{equation}
	\mathbf{I} 
	= \frac{\pi}{64} \cdot 10^{-8} \si{\m^4}
	\label{eqn:IWert}
\end{equation}
resultiert.

\subsection{Bestimmung des E-Moduls von Stab 1 mit beidseitiger Auflage}
\label{sec:messung1}
Bei Stab 1 ergaben sich bei beidseitiger Auflage mit einem Gewicht von $949.9 \si{\gram}$ 
die Messwerte in \autoref{tab:messdaten1}.

\begin{table}
	\centering
	\caption{Messdaten für einen beidseitig aufliegenden Stab, $x$ beschreibt
	die Position der Messuhr, wobei $0\si{\centi\meter}$
	dem rechten Auflagepunkt, $28.5\si{\centi\meter}$ der Mitte bzw. dem Auflagepunkt 
	des Gewichts entspricht und $57\si{\centi\meter}$ linken Auflagepunkt entspricht.
	$D$ beschreibt die Auslenkung des Stabes an der Stelle $x$ bei Belastung.}
	\label{tab:messdaten1}
	\sisetup{table-format=2.1}
	\begin{tabular}{c c c}
	\toprule
	% ueberschriften der einzelnen spalten}
	Messuhr &
	$x / \si{\centi\meter}$ &
	$D(x) / \si{\micro\meter}$
	\\
	\midrule
	1 & 47 & 200 \\
	1 & 20 & 200 \\
	1 & 23 & 310 \\
	1 & 9 & 95 \\
	1 & 13 & 160 \\
	1 & 15 & 185 \\
	2 & 40 & 285.5 \\
	2 & 35 & 360 \\
	2 & 31 & 380 \\
	2 & 42 & 275 \\
	2 & 38 & 210 \\
	2 & 64 & 130 \\
	\bottomrule
\end{tabular}
\end{table}

Der letzte Messwert (für $x = 64\si{\centi\meter}$) wird dabei in der Auswertung nicht
betrachtet, da dieser Außerhalb des Messbereichs liegt und somit keiner richtigen Messung 
zugeordnet werden kann.

Mit der Methode der kleinsten Quadrate wird die in \autoref{sec:Theorie} hergeleitete
Funktion 
\begin{equation}
	D(x) = \frac{F}{48 \, E \, \mathbf{I}} \left(4x^3 - 12Lx^2 + 9L^2x - L^3 \right)
\end{equation}
an die Messdaten gefittet. Da der Elastizitätsmodul der einzige
Freiheitsgrad des Polynoms ist, erhält man dadurch auch einen Wert für diesen.

Mit der Methode der kleinsten Quadrate kann somit eine Ausgleichkurve bestimmt
werden, wodurch man auch zu jeder Messreihe ein Wert für $E$ erhält.

Um die Abweichungen von der Theoriekurve besser Sichtbar zu machen, 
wurden in \autoref{fig:plot-messung1} die Messdaten gegen $3L^2x - 4x^3$ grafisch 
aufgetragen.
\begin{figure}[H]
	\centering
	\includegraphics{build/plot1.pdf}
	\caption{Messdaten und Theoriekurve zur Messung 1}
	\label{fig:plot-messung1}
\end{figure}

Diese Methode ergibt für den Elastizitätsmodul von Stab 1
\begin{equation}
	E = (192 \pm 13.1) \si{\giga\pascal}.
	\label{eqn:E-messung1}
\end{equation}
Wobei sich der Fehler aus der Covarianz-Matrix zum Curve-Fit ergibt.

\subsection{Bestimmung des E-Moduls von Stab 1 mit einseitiger Einspannung}
\label{sec:messung2}

Bei Stab 1 ergaben sich bei einseitiger Einspannung 
mit einem Gewicht von $949.9 \si{\gram}$ 
die Messwerte in \autoref{tab:messdaten2}.

\begin{table}
	\centering
	\caption{Messdaten für Stab 1, einseitig eingespannt. $x$ beschreibt
	die Position der Messuhr, wobei $0\si{\centi\meter}$
	dem Ort bei der Einspannung entspricht.
	$D$ beschreibt die Auslenkung des Stabes an der Stelle $x$ unter Last.}
	\label{tab:messdaten2}
	\sisetup{table-format=2.1}
	\begin{tabular}{c c c}
	\toprule
	% ueberschriften der einzelnen spalten}
	Messuhr &
	$x / \si{\centi\meter}$ &
	$D(x) / \si{\micro\meter}$
	\\
	\midrule
	% messuhr 1
	1 & 5 & 180 \\
	1 & 14 & 1120 \\
	1 & 18 & 2750 \\
	1 & 22 & 2500 \\
	1 & 26 & 3310 \\
	1 & 30 & 4200 \\
	1 & 34 & 5100 \\
	1 & 38 & 6010 \\
	1 & 16 & 1420 \\
	1 & 39 & 6010 \\
	% messuhr 2
	2 & 12 & 850 \\
	2 & 20 & 2100 \\
	2 & 24 & 2880 \\
	2 & 28 & 3800 \\
	2 & 32 & 4680 \\
	2 & 36 & 5600 \\
	2 & 40 & 6600 \\
	2 & 44 & 7600 \\
	2 & 42 & 7360 \\
	2 & 48 & 7500 \\
	\bottomrule
\end{tabular}
\end{table}

Wie in \autoref{sec:messung1} wird auch hier mit einer Ausgleichrechnung 
gearbeitet. Da es sich jetzt um eine einseitige Einspannung handelt wird mit der
Funktion
\begin{equation}
	D(x) = \frac{F}{2 \, E \, \mathbf{I}} \left(Lx^2 - \frac{x^3}{x} \right)
\end{equation}
gearbeitet.

Zur Verdeutlichung der Abweichungen werden in der grafischen Darstellung in
\autoref{fig:plot-messung2} die Messreihe und Theoriekurve gegen
$Lx^2 - x^3/3$ aufgetragen.
\begin{figure}[H]
	\centering
	\includegraphics{build/plot2.pdf}
	\caption{Messdaten und Theoriekurve zur Messung 2}
	\label{fig:plot-messung2}
\end{figure}

Mit dieser Methode erhält man ein Elastizitätsmodul für Stab 1 von
\begin{equation}
	E = (92.5 \pm 1.9) \si{\giga\pascal}
	\label{eqn:E-messung2}
\end{equation}

\subsection{Bestimmung des E-Moduls von Stab 2 mit einseitiger Einspannung}
\label{sec:messung3}

Da in der dritten Messreihe ebenfalls mit einem einseitig eingespannten Stab gearbeitet
wurde, ist das Vorgehen komplett analog zu dem in \autoref{sec:messung2}.
Auch hier wurde
der Stab am freien Ende mit einem Gewicht von $949.9\si{\gram}$ belastet.
\\
Die Messwerte sind in \autoref{tab:messdaten3} zu sehen.

\begin{table}
	\centering
	\caption{Messdaten für Stab 2, einseitig eingespannt. $x$ beschreibt
	die Position der Messuhr, wobei $0\si{\centi\meter}$
	dem Ort bei der Einspannung entspricht.
	$D$ beschreibt die Auslenkung des Stabes an der Stelle $x$ unter Last.}
	\label{tab:messdaten3}
	\sisetup{table-format=2.1}
	\begin{tabular}{c c c}
	\toprule
	% ueberschriften der einzelnen spalten}
	Messuhr &
	$x / \si{\centi\meter}$ &
	$D(x) / \si{\micro\meter}$
	\\
	\midrule
	% messuhr 1 \\
	1 & 4 & 90 \\
	1 & 6 & 175 \\
	1 & 8 & 280 \\
	1 & 16 & 965 \\
	1 & 18 & 1190 \\
	1 & 20 & 1420 \\
	1 & 28 & 2600 \\
	1 & 30 & 2870 \\
	1 & 32 & 3190 \\
	1 & 40 & 4480 \\
	% messuhr 2 \\
	2 & 10 & 410 \\
	2 & 12 & 570 \\
	2 & 14 & 760 \\
	2 & 22 & 1700 \\
	2 & 24 & 1990 \\
	2 & 26 & 2280 \\
	2 & 34 & 3500 \\
	2 & 36 & 3880 \\
	2 & 38 & 4200 \\
	2 & 46 & 5580 \\
	\bottomrule
\end{tabular}
\end{table}

Die Ausgleichsrechnung ergab dabei ein E-Modul zu Stab 2 von
\begin{equation}
	E = (130.9 \pm 0.9) \si{\giga\pascal}.
	\label{eqn:E-messung3}
\end{equation}

Ein Plot der Messdaten sowie eine Theoriekurve zum bestimmten E-Modul sind in 
\autoref{fig:plot-messung3} zu sehen. Die Unsicherheiten in der 
Covarianz-Matrix beim
Curve-Fit sind so klein, dass das Intervall kaum noch zu sehen ist.

\begin{figure}[H]
	\centering
	\includegraphics{build/plot3.pdf}
	\caption{Messdaten und Theoriekurve zur Messung 3}
	\label{fig:plot-messung3}
\end{figure}

% Machen wir nur evtl
% \subsection{Bestimmung der Dichte der Werkstoffe}
% \label{sec:Dichte}
