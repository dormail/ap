\section{Auswertung}
\label{sec:Auswertung}

\subsection{Messdaten}
\label{sec:Messdaten}
Es wurden insgesamt drei Messreihen an zwei Werksstoffen durchgeführt. Dabei
wurde ein Werkstoff einseitig eingespannt belastet und auch beidseitig aufliegend,
der zweite nur einseitig bespannt. Die Messungen liefen wie folgt ab:
\begin{enumerate}
	\item Werksstoff 1, beidseitig aufliegend
	\item Werksstoff 1, einseitg eingespannt 
	\item Werksstoff 2, einseitg eingespannt 
\end{enumerate}

Die Biegung wurde dabei immer mit zwei Uhren an zwei Orten gleichzeitig 
gemessen, welche zum Teil farblich getrennt in den Diagrammen abgetragen werden,
um Fehler in der Messaperatur zu erkennen.

\subsection{Nicht-lineare Ausgleichrechnung und Bestimmung des E-Moduls}
Wie in \autoref{sec:Theorie} hergeleitet, lässt sich die Biegung $D(x)$ mit 
Polynomen beschreiben. Bei einseitiger Einspannung gilt
\begin{equation}
	D(x) = \frac{F}{2 \, E \, \mathbf{I}} \left(Lx^2 - \frac{x^3}{3} \right).
	\label{eqn:d_einseitig}
\end{equation}

Bei beidseitiger Auflage gilt
\begin{equation}
	D(x) =
	\begin{cases}
		\frac{F}{48 \, E \, \mathbf{I}}  
		\left( 3L^2x - 4x^3 \right)
		& \text{falls } x \in \left[0, \frac L2 \right] \\
		\frac{F}{48 \, E \, \mathbf{I}}  
		\left( 4x^3 - 12Lx^2 + 9L^x - L^3 \right)
		& \text{falls } x \in \left[ \frac L2, L \right]
	\end{cases}
	.
	\label{eqn:d_beidseitig}
\end{equation}

Das Flächenträgheitsmoment $\mathbf{I}$ wird (da wir mit runden Stäben gearbeitet
haben) in Polarkoordinaten berechnet:
\begin{align}
	\mathbf{I} 
	&= \int_Q y^2 dq(y) 
	\\
	&= \int_0^{2\pi} d\varphi \int_0^r dr^\prime r^\prime 
	\left( r^\prime\sin(\varphi) \right)^2
	\\
	&= \frac{\pi}{4} r^4
\end{align}

Mit der Methode der kleinsten Quadrate kann somit eine Ausgleichkurve bestimmt
werden, wodurch man auch zu jeder Messreihe ein Wert für $E$ erhält.

\newpage
\begin{figure}[H]
	\includegraphics{build/plot1.pdf}
	\caption{Plot und Fit zu Messung 1}
	\label{fig:plot1}
\end{figure}
\begin{figure}[H]
	\includegraphics{build/plot2.pdf}
	\caption{Plot und Fit zu Messung 2}
	\label{fig:plot2}
\end{figure}
\subsubsection{Ausgleichsrechnung für Stab 1}
Da wir Stab 1 sowohl einseitig eingespannt, als auch beidseitig liegend,
getestet haben, entstehen mit dieser Methode zwei Ergebnisse zum E-Modul von diesem.

\begin{equation}
	E_1 = (192 \pm 13) \si{\giga\pascal},
	\label{eqn:E_messung1}
\end{equation}
\begin{equation}
	E_2 = (92.5 \pm 1.9) \si{\giga\pascal}
	\label{eqn:E_messung2}
\end{equation}
Dabei hat uns die Messung 1 einen Wert für $E$ geliefert
der sich stark von dem Ergebniss aus der zweiten Messung unterscheidet (vgl.
\ref{eqn:E_messung1} und \ref{eqn:E_messung2}).

Dazu fiel ein Messwert in der ersten Messung auf, wo für die Position
einer Messuhr
$64 \si{\cm}$ eingetragen wurde. Da die Messleiste nur bis $57 \si{\cm}$ geht
wurde dieser Wert in \autoref{fig:plot1} rot markiert und bei der Auswertung 
nicht mit einbezogen.

\subsubsection{Ausgleichsrechnung für Stab 2}
\begin{figure}[H]
	\includegraphics{build/plot3.pdf}
	\caption{Plot und Fit zu Messung 3}
	\label{fig:plot3}
\end{figure}

Für Stab 2 erhält man mit der selben Methode ein Elastizitätsmodul von
\begin{equation}
	E_3 = (130.9 \pm 0.96) \si{\giga\pascal}
	\label{eqn:E_messung3}
\end{equation}
Der Plot der Daten mit einer Theoriekurve für das bestimmte Elastizitätsmodul
ist in \autoref{fig:plot3} zu sehen.
