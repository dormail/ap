\section{Diskussion}
\label{sec:Diskussion}

\subsection{Diskussion der Ergebnisse für Stab 1}

Für Stab 1 wurden zwei Messreihen durchgeführt, diese ergaben zwei sehr 
unterschiedliche Werte von jeweils
\begin{align}
	E_\text{beid.} &= (192 \pm 13.1) \si{\giga\pascal} \\
	E_\text{eins.} &= (92.5 \pm 1.9) \si{\giga\pascal}.
\end{align}

Die Fehler liegen außerhalb einer statistischen Abweichung bei über $200\%$,
weshalb wir hier andere Fehler quellen betrachtet werden sollten.

Eine Ursache für die Differenz zwischen den Werten kann sein, dass die beidseitige
Auflage im Allgemeinen kleinere Auslenkungen enthielt, welche für größere 
Abweichungen beim Ablesen der Messuhren sorgen. Auch andere Quellen, die den absoluten
Fehler beeinflussen, üben dadurch mehr Einfluss auf die Messergebnisse aus.

Eine Fehlerquelle, welche die Messwerte absolut beeinflusst, ist die 
Sensitivität der Uhren. Gerade Messuhr 2 reagierte sehr stark auf kleine Vibrationen 
oder den wackelnden Tisch. Wir haben die Uhren vor jeder Messung auf 0 gestellt,
wobei dies nie perfekt möglich war, dabei hat das Zurücksetzen einer Uhr oft den 
den Messwert der anderen Uhr beeinflusst.
Dieser Fehler hätte dadurch eliminert werden können, indem man zum Beispiel nicht 

die Uhren selbst auf 0 stellt, sondern die Werte ohne und mit Last notiert, und
die Differenz berechnet.
\\
Ein Tisch der weniger wackelt hätte auch positive Auswirkungen auf den absoluten
Messfehler gehabt.
