\section{Diskussion}
\label{sec:Diskussion}

\subsection{Diskussion der Ergebnisse für Stab 1}
\label{sec:diskussion-stab1}

Für Stab 1 wurden zwei Messreihen durchgeführt, diese ergaben zwei sehr 
unterschiedliche Werte von jeweils
\begin{align}
	E_\text{beid.} &= (192 \pm 13.1) \si{\giga\pascal} \\
	E_\text{eins.} &= (92.5 \pm 1.9) \si{\giga\pascal}.
\end{align}

Die relative Abweichung zwischen den beiden Werten liegt bei über $200\%$,
weshalb hier Fehlerquellen in der Durchführung betrachtet werden sollten.

Eine Ursache für die Differenz zwischen den Werten kann sein, dass die 
beidseitige
Auflage im Allgemeinen kleinere Auslenkungen enthielt, welche für größere 
relative
Abweichungen beim Ablesen der Messuhren sorgen. Auch andere Quellen, die den
Fehler absolut beeinflussen, üben dadurch mehr Einfluss auf die Messergebnisse
aus.

Eine Fehlerquelle, welche die Messwerte absolut beeinflusst, ist die 
Sensitivität der Uhren. Gerade Messuhr 2 reagierte sehr stark auf kleine Vibrationen 
oder den wackelnden Tisch. Wir haben die Uhren vor jeder Messung auf 0 gestellt,
wobei dies nie perfekt möglich war, dabei hat das Zurücksetzen einer Uhr oft den 
den Messwert der anderen Uhr beeinflusst.

Dieser Fehler hätte dadurch eliminert werden können, indem man zum Beispiel nicht 
die Uhren selbst auf 0 stellt, sondern die Werte ohne und mit Last notiert, und
die Differenz berechnet.
\\
Ein Tisch der weniger wackelt hätte auch positive Auswirkungen auf den absoluten
Messfehler gehabt.

Auffällig für das Ergebnis bei Messung 1 ist auch eine hohe statistische 
Abweichung von $6.82\%$ vom Messwert (vgl. $1.99\%$ bei Messung 2). Diese lassen sich dadurch
erklären, dass Messung 1 als erstes durchgeführt wurde, wo nicht alle Einflüsse, wie zum Beispiel 
Vibrationen des Tisches, bekannt waren. Ab der zweiten Messung wurde verstärkt dadrauf geachtet, 
dass nach der Eichung der Uhren die Einflüsse auf die Messaperatur auf ein Minimum reduziert wurden,
was auch die geringere Varianz bei $E_\text{eins.}$ erklärt.

\subsection{Diskussion des Ergebnis für Stab 2}
\label{sec:diskussion-stab2}

Für Stab 2 wurde nur eine Messreihe durchgeführt, in welcher dieser einseitig eingespannt wurde.
Bei der Messreihe erhielten wir ein E-Modul von
\begin{equation}
	E = (130.9 \pm 0.9) \si{\giga\pascal}.
	\label{eqn:emodul2}
\end{equation}

Die Varianz ist dabei mit $0.69\%$ mit der aus Messung 2 vergleichbar. 
Das unterstützt auch unsere Vermutung in \autoref{sec:diskussion-stab1}, dass 
die starke Streuung in der ersten Messreihe Resultat von mangelhafter 
Sorgfalt ist.

Der Wert $E = 130.9 \si{\giga\pascal}$ ist dabei vergleichbar mit anderen Metallen, wie zum
Beispiel Kupfer ($E_\text{Kupfer} = 124 \si{\giga\pascal}$\cite{MaWi}).

Die Recherche ergab jedoch keinen Werkstoff, der genau diesem Wert entsprach. Grund für 
Abweichungen kann sein, dass der Stab schon ohne Last eine Biegung aufwies, wodurch ein 
Biegeverhalten resultiert, welches von dem in \autoref{sec:Theorie} hergeleiteten
abweichen kann.
