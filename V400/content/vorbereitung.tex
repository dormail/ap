% Vorbereitungsstuff
\section{Vorbereitung}
\label{sec:Vorbereitung}
Als Vorbereitung sollten verschihedene Brechungsindize aus passender Literatur
herausgesucht werden. Diese sind in \autoref{tab:brechungsindex} angegeben.
\begin{table}
	\centering
	\caption{Brechungsindex verschiedener Materialien. \cite{cosmos-indirekt}}
	\label{tab:brechungsindex}
	\sisetup{table-format=2.1}
	\begin{tabular}{c c}
		\toprule
		Material &
		Brechungsindex $n$ \\
		\midrule
		Luft 		& 1,000292 \\
		Wasser		& 1,33 \\
		Kronglas	& 1,46 ... 1,65 \\
		Plexiglas (PMMA)& 1,49 \\
		Diamant		& 2,42 \\
		\bottomrule
	\end{tabular}
\end{table}
Wegen der Unsicherheit für Kronglas wird mit $n(\text{Kronglas}) = 1,55$ gerechnet.
\\
Ferner sollte für die Liniendichten 600, 300, 100 Linien/mm die Gitterkonstante berechnet
werden. Dies ist der Kehrwert der Dichte, damit folgen
\begin{equation}
	d\left(100 \, \frac{\text{Linien}}{\si{\milli\meter}} \right) 
	= \SI{10}{\micro\meter},
	\qquad
	d\left(300 \, \frac{\text{Linien}}{\si{\milli\meter}} \right) 
	= \SI{3,3}{\micro\meter},
	\qquad
	d\left(600 \, \frac{\text{Linien}}{\si{\milli\meter}} \right) 
	= \SI{1,67}{\micro\meter}.
\end{equation}

