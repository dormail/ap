\section{Diskussion}
\label{sec:Diskussion}
Im ersten Teil wurde das Reflexionsgesetz überprüft. Das Verhältnis zwischen Einfall und
Ausfallwinkel, welches in der Theorie den Wert 1 hat, wurde experimentell mit dem
Ergebnis
\begin{equation}
	\frac{\alpha_1}{\alpha_2} = 1,013 \pm 0,016
\end{equation}
bestimmt. Die Abweichung liegt bei weniger als 2\% und der Theoriewert liegt innerhalb
einer Standartabweichung.
\\
Die Messung zum Brechungsgesetz verlief ähnlich gut. Es wurde ein Brechungsindex 
\begin{equation}
	n = 1,47 \pm 0,034
\end{equation}
experimentell nachgewiesen. Wie bei der ersten Messung, liegt der Literaturwerte für
Plexiglas aus PMMA 
\begin{equation}
	n_\text{Lit} = 1,49
\end{equation}
auch hier im Intervall einer Standartabweichung zum gemessenen Wert.
\\
Für das Prisma wurde der Ablenkwinkel $\delta$ berechnet. Die Ergebnisse sind in
\autoref{fig:plot4} grafisch dargestellt. Die errechneten Werte sind in 
\autoref{tab:disk-messwerte-prisma} nochmals zusammengefasst.
\begin{table}[H]
	\centering
	\caption{Zusammenhang zwischen Einfallswinkel $\alpha_1$ und Ablenkung $\delta$
	für rotes und grünes Licht.}
	\label{tab:disk-messwerte-prisma}
	\sisetup{table-format=2.1}
	\begin{tabular}{c c c}
		\toprule
		$\alpha_1$ &
		$\delta^\text{rot}$ &
		$\delta^\text{grün}$ \\
		\midrule
		30 & 49,2 & 50,0	\\
		35 & 42,5 & 43,1	\\
		40 & 39,7 & 40,3	\\
		45 & 38,8 & 39,3	\\
		50 & 38,2 & 38,5	\\
		\bottomrule
	\end{tabular}
\end{table}
\noindent
Im letzten Teil wurde Beugung am Gitter betrachtet. Dabei wurde mit den Winkeln der Maxima
bei drei verschiedenen Gittern die Wellenlänge des verwendeten Lasers berechnet. Die
einzelnen Messungen sind untereinander konsistent mit ihren Ergebnissen. Abschließend
wurden die Wellenlängen
\[
	\lambda^\text{rot} = (645 \pm 5) \si{\nano\meter}
	\qquad
	\lambda^\text{grün} = (538 \pm 6) \si{\nano\meter}
\] 
ermittelt, welche im Bereich der Literaturwerte für die entsprechenden Farbspektren liegen
\cite{frustfrei}.
\begin{equation}
	\lambda_\text{Lit}^\text{rot} = 650 \text{ bis } 750 \si{\nano\meter}
	\qquad
	\lambda_\text{Lit}^\text{grün} = 490 \text{ bis } 575 \si{\nano\meter}
\end{equation}

