\section{Auswertung}
\label{sec:Auswertung}
Im nachfolgenden Teil sollen die Messdaten ausgewertet werden. Dazu werden die Formeln für
Mittelwert und Standartabweichung
\begin{equation}
	\mu(T) = \bar{T}=\frac{1}{n}\sum_{\textrm{i=1}}^n T_\textrm{i}
	\label{eqn:mittelwert}
\end{equation}
\begin{equation}
	\sigma(T) = \sigma_T = \sqrt{\frac{\sum_{i=1}^{n}(T_i-\bar{T})^2}{n}}
	\label{eqn:standardabweichung}
\end{equation}
\noindent
verwendet. Die beiden Formeln beziehen sich jeweils auf $n$ Messwerte.

\subsection{Reflexionsgesetz}
\label{sec:Reflexionsgesetz}
In diesem Teil soll das Reflexionsgesetz $\alpha_1 = \alpha_2$ überprüft werden. Es folgt
direkt das Verhältnis 
\begin{equation}
	\frac{\alpha_1}{\alpha_2} = 1.
\end{equation}
Zu sechs verschiedenen Einfallswinkeln wurde der Ausfallwinkel gemessen, die Messdaten
sowie das Winkelverhältnis
sind in \autoref{tab:messwerte-reflexionsgesetz} angegeben.
\begin{table}
	\centering
	\caption{Messwerte und Winkelverhältnis für das Reflexionsgesetz. Die Winkel sind
	in Grad.}
	\label{tab:messwerte-reflexionsgesetz}
	\sisetup{table-format=2.1}
	\begin{tabular}{c c c}
		\toprule
		$\alpha_1$ & $\alpha_2$ & $\frac{\alpha_1}{\alpha_2}$ \\
		\midrule
		20 & 19,5 & 0,975 \\
		25 & 25,5 & 1,02  \\
		29 & 29,5 & 1,017 \\
		35 & 35,5 & 1,014 \\
		40 & 41   & 1,025 \\
		45 & 46   & 1,022 \\
		50 & 51   & 1,02  \\
		\bottomrule
	\end{tabular}
\end{table}
Aus den Werten für das Verhältnis lassen sich Mittelwert und Standartabweichung berechnen
\begin{equation}
	\mu\left(\frac{\alpha_1}{\alpha_2}\right) = 1,013 
	\qquad
	\sigma\left(\frac{\alpha_1}{\alpha_2}\right) = 0,016.
	\label{eqn:ergebnis1}
\end{equation}
Damit folgt das experimentell bestimmte Winkelverhältnis von
\begin{equation}
	\frac{\alpha_1}{\alpha_2} = 1,013 \pm 0,016.
\end{equation}
In \autoref{fig:plot1} sind die Messdaten grafisch dargestellt.
\begin{figure}[H]
	\centering
	\includegraphics{build/1.pdf}
	\caption{Winkelverhältnis aus den Messdaten sowie Mittelwert, Standartabweichung
	und Theoriewert $1$.}
	\label{fig:plot1}
\end{figure}

\subsection{Brechungsgesetz}
\label{sec:Brechungsgesetz}
Nun soll das Brechungsgesetz
\begin{equation}
	\frac{\sin\alpha}{\sin\beta} = n
	\label{eqn:ausw:brechungsgesetz}
\end{equation}
untersucht werden. Das vorgehen ist ähnlich zu dem für das Reflexionsgesetz im vorherigen
Abschnitt. Hier ist jedoch das Verhältnis des sinus zu Ein- und Brechungswinkel von
Interesse. Die Messwerte, der sinus der Winkel sowie das Verhältnis der beiden Sinuswerte
sind in \autoref{tab:messwerte-brechungssgesetz} gegeben.
\begin{table}
	\centering
	\caption{Messwerte und Winkelverhältnis für das Brechungsgesetz. Die Winkel sind
	in Grad.}
	\label{tab:messwerte-brechungssgesetz}
	\sisetup{table-format=2.1}
	\begin{tabular}{c c c c c}
		\toprule
		$\alpha$ &
		$\beta$ &
		$\sin\alpha$ &
		$\sin\beta$ &
		$\frac{\sin\alpha}{\sin\beta}$ \\
		\midrule
		30 & 19   & 0,500 & 0,326 & 1,536	\\
		35 & 24   & 0,574 & 0,407 & 1,410	\\
		40 & 26   & 0,643 & 0,438 & 1,466	\\
		50 & 31   & 0,766 & 0,515 & 1,487	\\
		55 & 34   & 0,819 & 0,559 & 1,465	\\
		60 & 36   & 0,866 & 0,588 & 1,473	\\
		70 & 39,5 & 0,940 & 0,636 & 1,477	\\
		\bottomrule
	\end{tabular}
\end{table}
Das weitere Vorgehen ist ebenso analog. Mittelwert und Standartabweichung folgen aus den
oben genannten Gleichungen, mit diesen lässt sich hier
für den Brechungsindex
\begin{equation}
	n = 1,47 \pm 0,034
	\label{eqn:brechung-exp}
\end{equation}
errechnen. Als Theoriewert folgt aus der Literatur \cite{cosmos-indirekt}
\begin{equation}
	n_\text{Theo} = 1,49.
	\label{eqn:brechung-theo}
\end{equation}
Die für die Rechnung verwendeten Werte sind auch nochmals in \autoref{fig:plot2} dargestellt.
\begin{figure}[H]
	\centering
	\includegraphics{build/2.pdf}
	\caption{Verhältnis der  Werte von Einfalls- und Brechungswinkel.}
	\label{fig:plot2}
\end{figure}

\subsection{Strahlversatz}
\label{sec:Strahlversatz}
Für die in \autoref{sec:Brechungsgesetz} gemessenen Werte lässt sich auch der Strahlversatz
berechnen. Für planparallele Platten ist dieser gegeben durch
\begin{equation}
	s = d \frac{\sin(\alpha - \beta)}{\cos{\beta}}.
	\label{eqn:ausw:versatz}
\end{equation}
Dabei ist $d = \SI{5.85}{\centi\meter}$ die Dicke der Platte. Die errechneten Werte sind
in \autoref{tab:strahlversatz} angegeben.
\begin{table}
	\centering
	\caption{Strahlversatz für die gemessenen Winkelpaare $\alpha$ und $\beta$.}
	\label{tab:strahlversatz}
	\sisetup{table-format=2.1}
	\begin{tabular}{c c c}
		\toprule
		$\alpha$ &
		$\beta$ &
		$s / \si{\centi\meter}$ \\
		\midrule
		30 & 19   &  3,4	\\
		35 & 24   &  2,7	\\
		40 & 26   &  3,2	\\
		50 & 31   &  3,7	\\
		55 & 34   &  3,7	\\
		60 & 36   &  4	\\
		70 & 39,5 &  4,7	\\
		\bottomrule
	\end{tabular}
\end{table}

\subsection{Prisma}
\label{sec:Prisma}
In diesem Teil soll nun die Ablenkung $\delta$ von einem Prisma für zwei Wellenlängen
bestimmt werden. Mit der Formel
\begin{equation}
	\delta = (\alpha_1 + \alpha2) - (\beta_1 + \beta_2)
\end{equation}
und dem Zusammenhang
\begin{equation}
	\beta_i = \arcsin\left(\frac{\sin\alpha_i}{n}\right),
\end{equation}
welcher direkt aus dem Snelliusschen Brechungsgesetz folgt, lässt sich $\delta$ bestimmen.
Die Messwerte sowie die Ergebnisse für die Ablenkung sind in \autoref{tab:messwerte-prisma}
dargestellt. Die Abhängigkeit $\delta(\alpha_1)$ ist auch in \autoref{fig:plot4} grafisch
dargestellt.
\begin{table}
	\centering
	\caption{Messwerte und errechnete Werte für das Prisma.}
	\label{tab:messwerte-prisma}
	\sisetup{table-format=2.1}
	\begin{tabular}{c c c c c c c c}
		\toprule
		$\alpha_1$ &
		$\alpha_2^\text{rot}$ &
		$\alpha_2^\text{grün}$ &
		$\beta_1$ &
		$\beta_2^\text{rot}$ &
		$\beta_2^\text{grün}$ &
		$\delta^\text{rot}$ &
		$\delta^\text{grün}$ \\
		\midrule
		30 & 77 & 78   & 18,8 & 38,9 & 39,1 & 49,2 & 50,0	\\
		35 & 65 & 66   & 21,7 & 35,8 & 36,1 & 42,5 & 43,1	\\
		40 & 57 & 58   & 24,5 & 32,8 & 33,2 & 39,7 & 40,3	\\
		45 & 51 & 52   & 27,1 & 30,1 & 30,6 & 38,8 & 39,3	\\
		50 & 45 & 45,5 & 29,6 & 27,1 & 27,4 & 38,2 & 38,5	\\
		\bottomrule
	\end{tabular}
\end{table}
\begin{figure}[H]
	\centering
	\includegraphics{build/4.pdf}
	\caption{Die errechneten Werte der Ablenkung in Abhängigkeit vom eingestellten
	Einfallwinkel $\alpha_1$.}
	\label{fig:plot4}
\end{figure}

\subsection{Beugung am Gitter}
\label{sec:Beugung am Gitter}
Für den letzten Teil wurden drei Gitter mit $100$, $300$ bzw. $600
\,\text{Linien}/\si{\milli\meter}$ betrachtet. Die Gitterkonstanten sind jeweils die
Kehrwerte mit
\begin{equation}
	d\left(100 \, \frac{\text{Linien}}{\si{\milli\meter}} \right) 
	= \SI{10}{\micro\meter}
	\qquad
	d\left(300 \, \frac{\text{Linien}}{\si{\milli\meter}} \right) 
	= \SI{3,3}{\micro\meter}
	\qquad
	d\left(600 \, \frac{\text{Linien}}{\si{\milli\meter}} \right) 
	= \SI{1,67}{\micro\meter}.
\end{equation}
Die Messwerte sind in den nachfolgenden Tabellen gegeben. Es wurde jeweils nach der linken
und rechten Streuung getrennt, wodurch für jeden Lasen und jede Ordnung je zwei Winkel
$\varphi_i$ gemessen wurden.
\begin{table}
	\centering
	\caption{Messwerte zu Streuung an einem 100 Linien/mm Gitter. Die Winkel sind in
	Grad}
	\label{tab:messwerte-gitter100}
	\sisetup{table-format=2.1}
	\begin{tabular}{c c c c c}
		\toprule
		Beugungsordnung $k$ &
		$\varphi_1^\text{grün}$ &
		$\varphi_2^\text{grün}$ &
		$\varphi_1^\text{rot}$ &
		$\varphi_2^\text{rot}$ \\
		\midrule
		1 & 3    & 3,2  &  3,8 & 3,8 \\
		2 & 6    & 6,2  &  7,2 & 7,5 \\
		3 & 9,1  & 9,5  & 11,1 & 11,1 \\
		4 & 12,2 & 12,6 & 15   & 15 \\
		5 & 15,5 & 16 \\
		6 & 18,8 & 19,2 \\
		7 & 22   & 22,8 \\
		\bottomrule
	\end{tabular}
\end{table}
\begin{table}
	\centering
	\caption{Messwerte zu Streuung an einem 300 Linien/mm Gitter. Die Winkel sind in
	Grad}
	\label{tab:messwerte-gitter300}
	\sisetup{table-format=2.1}
	\begin{tabular}{c c c c c}
		\toprule
		Beugungsordnung $k$ &
		$\varphi_1^\text{grün}$ &
		$\varphi_2^\text{grün}$ &
		$\varphi_1^\text{rot}$ &
		$\varphi_2^\text{rot}$ \\
		\midrule
		1 & 9    & 9,3  &  11   & 11 \\
		2 & 18,2 & 19   &  22,2 & 22,8 \\
		\bottomrule
	\end{tabular}
\end{table}
\begin{table}
	\centering
	\caption{Messwerte zu Streuung an einem 600 Linien/mm Gitter. Die Winkel sind in
	Grad}
	\label{tab:messwerte-gitter600}
	\sisetup{table-format=2.1}
	\begin{tabular}{c c c c c}
		\toprule
		Beugungsordnung $k$ &
		$\varphi_1^\text{grün}$ &
		$\varphi_2^\text{grün}$ &
		$\varphi_1^\text{rot}$ &
		$\varphi_2^\text{rot}$ \\
		\midrule
		1 & 18,8 & 19,5 & 23,5 & 22,8 \\
		\bottomrule
	\end{tabular}
\end{table}
Für die Auswertung wurde zunächst mit der Formel
\begin{equation}
	\lambda = d \frac{\sin\varphi}{k}
	\label{eqn:ausw:beugungsmaxima}
\end{equation}
mit den jeweiligen Werten für $d$ für jeden Winkel die Wellenlänge bestimmt. Zuletzt
wurde von diesen Wellenlängen Mittelwert und Standartabweichung berechnet. Das ergab die
Werte in \autoref{tab:ergebnisse5}.
\begin{table}
	\centering
	\caption{Ergebnisse für die Wellenlängen der zwei Laser.}
	\label{tab:ergebnisse5}
	\sisetup{table-format=2.1}
	\begin{tabular}{c c c}
		\toprule
		&
		$\lambda^\text{rot}$ &
		$\lambda^\text{grün}$ \\

		$100 \, \text{Linien} / \si{\milli\meter}$ &
		(648 \pm 11,1) \si{\nano\meter} &
		(540 \pm 11,2) \si{\nano\meter} \\
		$300 \, \text{Linien} / \si{\milli\meter}$ & 
		(631 \pm 5,7) \si{\nano\meter} &
		(526 \pm 9,8) \si{\nano\meter} \\
		$600 \, \text{Linien} / \si{\milli\meter}$ & 
		(656 \pm 9,4) \si{\nano\meter} &
		(548 \pm 9,6) \si{\nano\meter} \\
		\bottomrule
	\end{tabular}
\end{table}
Zuletzt kann mit der Python Bibliothek uncertainties über die in \autoref{tab:ergebnisse5}
angegeben Werte gemittelt werden. Mit dieser erhält man die Wellenlängen
\begin{equation}
	\lambda^\text{rot} = (645 \pm 5) \si{\nano\meter}
	\qquad
	\lambda^\text{grün} = (538 \pm 6) \si{\nano\meter}
\end{equation}

