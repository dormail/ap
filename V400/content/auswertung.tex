\section{Auswertung}
\label{sec:Auswertung}
Im nachfolgenden Teil sollen die Messdaten ausgewertet werden. Dazu werden die Formeln für
Mittelwert und Standartabweichung
\begin{equation}
	\mu(T) = \bar{T}=\frac{1}{n}\sum_{\textrm{i=1}}^n T_\textrm{i}
    \label{eqn:mittelwert}
\end{equation}
\begin{equation}
	\sigma(T) = \sigma_T = \sqrt{\frac{\sum_{i=1}^{n}(T_i-\bar{T})^2}{n}}
    \label{eqn:standardabweichung}
\end{equation}
\noindent
verwendet. Die beiden Formeln beziehen sich jeweils auf $n$ Messwerte.

\subsection{Reflexionsgesetz}
\label{sec:Reflexionsgesetz}
In diesem Teil soll das Reflexionsgesetz $\alpha_1 = \alpha_2$ überprüft werden. Es folgt
direkt das Verhältnis 
\begin{equation}
	\frac{\alpha_1}{\alpha_2} = 1.
\end{equation}
Zu sechs verschiedenen Einfallswinkeln wurde der Ausfallwinkel gemessen, die Messdaten
sowie das Winkelverhältnis
sind in \autoref{tab:messwerte-reflexionsgesetz} angegeben.
\begin{table}
	\centering
	\caption{Messwerte und Winkelverhältnis für das Reflexionsgesetz. Die Winkel sind
	in Grad.}
	\label{tab:messwerte-reflexionsgesetz}
	\sisetup{table-format=2.1}
	\begin{tabular}{c c c}
		\toprule
		$\alpha_1$ & $\alpha_2$ & $\frac{\alpha_1}{\alpha_2}$ \\
		\midrule
		20 & 19,5 & 0,975 \\
		25 & 25,5 & 1,02  \\
		29 & 29,5 & 1,017 \\
		35 & 35,5 & 1,014 \\
		40 & 41   & 1,025 \\
		45 & 46   & 1,022 \\
		50 & 51   & 1,02  \\
		\bottomrule
	\end{tabular}
\end{table}
Aus den Werten für das Verhältnis lassen sich Mittelwert und Standartabweichung berechnen
\begin{equation}
	\mu\left(\frac{\alpha_1}{\alpha_2}\right) = 1,013 
	\qquad
	\sigma\left(\frac{\alpha_1}{\alpha_2}\right) = 0,016.
	\label{eqn:ergebnis1}
\end{equation}
Damit folgt das experimentell bestimmte Winkelverhältnis von
\begin{equation}
	\frac{\alpha_1}{\alpha_2} = 1,013 \pm 0,016.
\end{equation}
In \autoref{fig:plot1} sind die Messdaten grafisch dargestellt.
\begin{figure}[H]
	\centering
	\includegraphics{build/1.pdf}
	\caption{Winkelverhältnis aus den Messdaten sowie Mittelwert, Standartabweichung
	und Theoriewert $1$.}
	\label{fig:plot1}
\end{figure}

\subsection{Brechungsgesetz}
\label{sec:Brechungsgesetz}
Nun soll das Brechungsgesetz
\begin{equation}
	\frac{\sin\alpha}{\sin\beta} = n
	\label{eqn:ausw:brechungsgesetz}
\end{equation}
untersucht werden. Das vorgehen ist ähnlich zu dem für das Reflexionsgesetz im vorherigen
Abschnitt. Hier ist jedoch das Verhältnis des sinus zu Ein- und Brechungswinkel von
Interesse. Die Messwerte, der sinus der Winkel sowie das Verhältnis der beiden Sinuswerte
sind in \autoref{tab:messwerte-brechungssgesetz} gegeben.
\begin{table}
	\centering
	\caption{Messwerte und Winkelverhältnis für das Brechungsgesetz. Die Winkel sind
	in Grad.}
	\label{tab:messwerte-brechungssgesetz}
	\sisetup{table-format=2.1}
	\begin{tabular}{c c c c c}
		\toprule
		$\alpha$ &
		$\beta$ &
		$\sin\alpha$ &
		$\sin\beta$ &
		$\frac{\sin\alpha}{\sin\beta}$ \\
		\midrule
		30 & 19   & 0,500 & 0,326 & 1,536	\\
		35 & 24   & 0,574 & 0,407 & 1,410	\\
		40 & 26   & 0,643 & 0,438 & 1,466	\\
		50 & 31   & 0,766 & 0,515 & 1,487	\\
		55 & 34   & 0,819 & 0,559 & 1,465	\\
		60 & 36   & 0,866 & 0,588 & 1,473	\\
		70 & 39,5 & 0,940 & 0,636 & 1,477	\\
		\bottomrule
	\end{tabular}
\end{table}
Das weitere Vorgehen ist ebenso analog. Mittelwert und Standartabweichung folgen gemäß
\autoref{eqn:mittelwert} und \autoref{eqn:standardabweichung}, mit diesen lässt sich hier
für der Brechungsindex
\begin{equation}
	n = 1,47 \pm 0,034
	\label{eqn:brechung-exp}
\end{equation}
errechnen. Als Theoriewert folgt aus der Literatur
\begin{equation}
	n_\text{Theo} = 1,49.
	\label{eqn:brechung-theo}
\end{equation}
Die für die Rechnung verwendeten Werte sind auch nochmals in \autoref{fig:plot2} dargestellt.
\begin{figure}[H]
	\centering
	\includegraphics{build/2.pdf}
	\caption{Verhältnis der sinus Werte von Einfalls- und Brechungswinkel. Der
	Theoriewert gemäß \autoref{eqn:brechung-theo}.}
	\label{fig:plot2}
\end{figure}
